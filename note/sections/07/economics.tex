\input{sections/07/economics_macros}
\input{sections/07/economics_calculations}

\section{Технико-экономическое обоснование разработки и внедрения программного средства}
\label{sec:economics}

\subsection{Характеристика программного средства}
\label{sec:economics:description}

Разрабатываемый программное средство создания веб-приложений с помощью готовых графических компонентов представляет собой интегрируемый программный модуль. Разрабатываемый модуль позволяет:
\begin{itemize}
	\item чистая дисконтированная стоимость (ЧДД);
	\item срок окупаемости инвестиций (ТОК);
	\item рентабельность инвестиций (Ри).
\end{itemize}

Разработка программного модуля осуществлялась на языке JavaScript с использованием технологии Webix. 

Разработанное программное средство предполагает, что его интеграция будет производиться в приложение, построенное на основе архитектуры клиент ‒ сервер, где сервером является web-сервер, а клиентом приложение, в которое будет интегрироваться данный программное средство. При инициализации программный модуль будет отправлять запрос на сервер с целью получения конфигураций базовых компонентов, которые и будут использоваться конечным пользователем при создании своего макета приложения. 

Интегрируемый программный модуль позволяет разработчику приложения, являющегося местом интеграции, дать пользователю возможность самому определить внешний вид приложения согласно своим предпочтениям. Набор базовых компонентов и правила создания макета определяет разработчик, в чей продукт будет интегрироваться данное программное средство.

Разрабатываемый программный модуль имеет следующие
преимущества:
\begin{itemize}
	\item простой и понятный интерфейс;
	\item легко конфигурируется;
	\item гибкая интеграция.
\end{itemize}

\subsection{Расчет затрат на разработку и отпускной цены программного модуля}
\label{sec:economics:labouriousness}

Общий объем программного модуля определяется на основе информации о функциях разрабатываемого программного модуля, исходя из количества и объема функций, реализуемых программным модулем, по формуле:

\begin{equation}
	\totalloc = \sum_{i=1}^{n} \text{V}_i,
\end{equation}
\begin{explanation}
	где & $ \text{V}_i $ & объем i-ой функции программного модуля (количество строк исходного кода (LОС));\\
	& $ n $ & общее число функций.
\end{explanation}

С учетом условий разработки общий объем программного модуля уточняется в организации и определяется уточненный объем программного модуля по формуле:
\begin{equation}
	\text{V}_y = \sum_{i=1}^{n} \text{V}_\text{yi},
\end{equation}
\begin{explanation}
	где & $ \text{V}_\text{yi} $ & уточненный объем i-й функции программного модуля (LОС).
\end{explanation}

Расчет объема функций программного средства и общего объема приведен в таблице~\ref{table:economics:labouriousness:function_sizes}.

\begin{table}[!ht]
\caption{Перечень и объём функций программного модуля}
\label{table:economics:labouriousness:function_sizes}
\centering
	\begin{tabular}{{ | >{\centering}m{0.12\textwidth} | 
	>{\raggedright}m{0.6\textwidth} | 
	>{\centering\arraybackslash}m{0.2\textwidth}|}}

  	\hline
	\No{} функции & 
	{\begin{center} Наименование (содержание) \end{center}} & 
	Объём функции, LoC \\
  
	\hline 
	101 & Организация ввода информации & \num{300} \\

	\hline
	102 & Контроль, предварительная обработка и ввод информации & \num{500} \\

	\hline
	109 & Организация ввода/вывода информации в интерактивном режиме & \num{190} \\

	\hline
	111 & Управление вводом/выводом & \num{600} \\

	\hline
	207 & Манипулирование данными & \num{4000} \\

	\hline
	506 & Обработка ошибочных и сбойных ситуаций & \num{500} \\

	\hline
	507 & Обеспечение интерфейса между компонентами & \num{750} \\

	\hline
	601 & Отладка прикладных программ в интерактивном режиме & \num{1300} \\

	\hline
	707 & Графический вывод результатов & \num{4500} \\

	\hline
	 & Общий объем & \totallocfactor \\

	\hline
	\end{tabular}
\end{table}

В	связи с использованием более совершенных средств автоматизации объемы функций были уменьшены и уточненный объем программного модуля составил 10500 LОС вместо 11640 LОС.

Программный модуль относится ко второй категории сложности, и, следовательно, нормативная трудоемкость составит $\humandaysamount$ чел./дн.

Коэффициенты использования стандартных модулей и новизны программного модуля, составят $\stdmodules$ = $\num{0.06}$ и $\novelty$ = $\num{0.07}$ соответственно.

Коэффициент сложности программного модуля $ \complexity $ рассчитывается по формуле:
\begin{equation}
	\complexity = 1 + \sum_{i=1}^{n} \text{К}_i = 1 + \num{0.06} + \num{0.07} = \complexityfactor,
\end{equation}
\begin{explanation}
где & $ \text{К}_i $ & нормативная трудоемкость разработки программного модуля (\num{78} чел./дн.);\\
& $ n $ & количество учитываемых характеристик.
\end{explanation}

Нормативная трудоемкость служит основой для определения общей трудоемкости разработки программного модуля, который определяется по формуле:
\begin{equation}
	\totallaboriousness = \normativelaboriousness \cdot \complexity \cdot \stdmodules \cdot \novelty = \humandaysamount \cdot \stdmodulesvalue \cdot \noveltyvalue \cdot \complexityfactor = \totallaboriousnessvalue~\text{чел./дн.},
\end{equation}
\begin{explanation}
где & $ \normativelaboriousness $ & коэффициент, соответствующий степени повышения сложности за счет конкретной характеристики;\\
& $ \stdmodules $ & поправочный коэффициент, учитывающий степень использования при разработке стандартных модулей;\\
& $ \novelty $ & коэффициент, учитывающий степень новизны программного модуля;\\
& $ \complexity $ & коэффициент, учитывающий сложность программного модуля.
\end{explanation}

В соответствии с договором срок разработки ‒ \num{1.5} месяца (\num{0.16} г.).

Эффективный фонд времени работы одного человека рассчитывается по формуле:
\begin{equation}
	\effectivetimefund = \daysinyear - \holidays - \weekends - \vacationdays = \num{\daysInYear} - \num{\redLettersDaysInYear} - \num{103} - \num{24} = \num{229}~\text{дн.},
\end{equation}
\begin{explanation}
где & $ \text{Д}_\text{г} $ & количество дней в году;\\
& $ \text{Д}_\text{п} $ & количество праздничных дней в году;\\
& $ \text{Д}_\text{в} $ & количество выходных дней в году;\\
& $ \text{Д}_\text{о} $ & количество дней отпуска.
\end{explanation}

Численность исполнителей проекта($\executorsamount$) рассчитывается по формуле:
\begin{equation}
	\developersnumber = \frac{\totallaboriousness}{\developmenttime \cdot \effectivetimefund} = \frac{\num{49}}{\num{0,16} \cdot \num{229}} = \num{2}~\text{чел.},
\end{equation}
\begin{explanation}
где & $\totallaboriousness$ & общая трудоемкость разработки проекта, чел./дн.;\\
& $\developmenttime$ & срок разработки проекта, лет;\\
& $\effectivetimefund$ & эффективный фонд времени работы одного человека.
\end{explanation}

Разработкой программного средства создания веб-приложений с помощью готовых графических компонентов занимаются двое исполнителей: фронт-энд разработчик и тестировщик.

\subsection{Расчет сметы затрат}
\label{sec:economics:estimate}

Расчет основной заработной платы исполнителей представлен в таблице~\ref{table:economics:estimate:employees}

\begin{table}[!ht]
  \caption{Работники, занятые в проекте}
  \label{table:economics:estimate:employees}
  \begin{tabular}{| >{\raggedright}m{0.34\textwidth} 
                  | >{\centering}m{0.2\textwidth}
                  | >{\centering}m{0.2\textwidth}
                  | >{\centering\arraybackslash}m{0.15\textwidth}|}
	\hline
	{\begin{center}Исполнители\end{center}} & Эффективный фонд времени работы, дней & Дневная тарифная ставка, \byn & Месячный оклад, \byn\\

	\hline
	Фронт-энд разработчик & 35 & 90 & \num{3150} \\

	\hline
	Тестировщик & 20 & 60 & \num{1200}\\

	\hline
	Всего & 55 & & 4350\\

	\hline
	Премия (30\%) & & & 1305 \\

	\hline
	Основная заработная плата & & & 5655\\
	
	\hline
  \end{tabular}
\end{table}

Дополнительная заработная плата включает выплаты, предусмотренные законодательство о труде: оплата отпусков, льготных часов, времени  выполнения государственных обязанностей и других выплат, не связанных с основной деятельностью исполнителей, и определяется по нормативу, установленному в организации, в процентах к основной заработной плате.
Приняв данный норматив $\additionalwageratesymbol = \additionalwageratevalue$, рассчитаем дополнительные выплаты
\begin{equation}
	\additionalwagesymbol = \frac{\basewagesymbol \cdot \additionalwageratesymbol}{100\%} = \frac{\num{5655} \cdot \num{10}\%}{100\%} = \num{565.5}~\byn,
\end{equation}
\begin{explanation}
	где & $ \basewagesymbol $ & норматив дополнительной заработной платы (10\%).
\end{explanation}

Отчисления в фонд социальной защиты населения и в фонд обязательного страхования определяются в соответствии с действующим законодательством по нормативу в процентном отношении к фонду основной и дополнительной зарплат по следующим формулам:
\begin{equation}
	\ssfchargessymbol = \frac{(\basewagesymbol + \additionalwagesymbol) \cdot \ssfratesymbol}{100\%} = \frac{(5655 + \num{565.5}) \cdot \num{34.6}\%}{100\%} = \num{2152.3}~\byn,
\end{equation}
\begin{explanation}
	где & $ \basewagesymbol $ & норматив дополнительной заработной платы (10\%);\\
	& $\additionalwagesymbol$ & дополнительная заработная плата, \byn;\\
	& $\ssfratesymbol$ & норматив отчислений в фонд социальной защиты населения и на обязательное страхование (34 + \num{0.6}\%).
\end{explanation}

Расходы по статье <<Материалы>> рассчитываются по формуле:
\begin{equation}
	\consumableschargessymbol = \text{Н}_\text{м} \cdot \frac{\text{V}_\text{o}}{100} = \num{0.5} \cdot \frac{10500}{100} = \num{52.5}~\byn,
\end{equation}
\begin{explanation}
	где & $\text{Н}_\text{м}$ & норма расхода материалов в расчете на 100 строк исходного кода программного модуля (\num{0.5}~\byn);\\
	& $\text{V}_\text{o}$ & общий объем программного модуля (10500 LoC).
\end{explanation}

Расходы по статье <<Машинное время>> включают оплату машинного времени, необходимого для разработки и отладки ПО, которое определяется по нормативам на 100 строк исходного кода и рассчитываются по формуле:
\begin{equation}
	\machinetimechargessymbol = \machinehourpricesymbol \cdot \frac{\totalloc}{100} \cdot \machinetimeratesymbol = \num{0.3} \cdot \frac{\totallocfactor}{100} \cdot 12 = \num{419}~\byn,
\end{equation}
\begin{explanation}
где & $\machinehourpricesymbol$ & цена одного машино-часа (\num{0.3}~\byn);\\
& $\machinetimeratesymbol$ & норматив расхода машинного времени на отладку 100 строк исходного кода (12 машино-часов).
\end{explanation}

Затраты по статье <<Накладные расходы>>, связанные с необходимостью  содержания  аппарата  управления,  вспомогательных хозяйств и опытных (экспериментальных) производств, а также с расходами на общехозяйственные нужды, относятся к конкретному ПО по нормативу в процентном отношении к основной заработной плате
исполнителей. Принимая норматив равным $\overheadratesymbol~=~\num{80}\%$ получим величину расходов
\begin{equation}
	\overheadchargessymbol = \frac{\basewagesymbol \cdot \overheadratesymbol}{100\%} = \frac{\num{5655} \cdot 80\%}{100\%} = \num{4524}~\byn
\end{equation}

Общая сумма расходов по смете определяется как сумма вышерассчитанных показателей и рассчитывается по формуле:
\begin{equation}
\begin{aligned}
	\totalchargessymbol = \basewagesymbol + \additionalwagesymbol + \ssfchargessymbol &+ \insurancechargessymbol + \consumableschargessymbol + \machinetimechargessymbol + \businesstripchargessymbol + \otherchargessymbol + \overheadchargessymbol =\\
	&=\num{13368.3}~\byn
\end{aligned}
\end{equation}

\subsection{Расчет экономической эффективности реализации на рынке автоматизированной системы управления курьерской службой}
\label{sec:economics:profit}

Программный модуль планируется разместить в сети Интернет, организации, имеющие в структуре курьерскую службу доставки грузов, смогут скачать его по цене, определенной на основе маркетинговых исследований рынка систем управления контентом сайта, 2400 руб.

В	среднем в год прогнозируется, что 10 организаций скачает разработанный программный модуль.

Налог на добавленную стоимость составит:
\begin{equation}
	\text{НДС} = \frac{24000 \cdot 20\%}{100\%} = 4800~\byn.
\end{equation}

Таким образом, разработчик получит доход в размере 24000 руб. Прирост чистой прибыли от реализации программного модуля в сети Интернет можно определить по формуле~\ref{eq:economics:profit:pureprofit}.
В	первый год реализации программного модуля на рынке разработчик получит чистую прибыль в размере:: 
\begin{equation}
\label{eq:economics:profit:pureprofit}
\begin{aligned}
	\Delta\text{П} = (&\text{Д} - \text{НДС} - \text{З}) \cdot (\num{1} - \text{H}_\text{п}) =\\
	=(24000 - 4800 - &\num{13368.3}) \cdot (1 - \num{0.18}) = \num{4198.8}~\byn,
\end{aligned}
\end{equation}
\begin{explanation}
где & $\text{Д}$ & доход полученный от реализации продукта, \byn;\\
& $\text{НДС}$ & налог на добавленную стоимость (20\%);\\
& $\text{З}$ & затраты на разработку программного модуля, \byn;\\
& $\text{H}_\text{п}$ & ставка налога на прибыль (18\%).
\end{explanation}

Рентабельность затрат на разработку программного средства создания веб-приложений с помощью готовых графических компонентов составит:
\begin{equation}
	\text{P} = \frac{\Delta\text{П}}{\totalchargessymbol} \cdot \num{100}\% = \frac{\num{4198.8}}{\num{13368.3}} \cdot \num{100}\% = \num{31.4}\%,
\end{equation}

Таким образом, все затраты на разработку программного модуля окупятся в первый год от реализации программного модуля в сети Интернет. Преимущества разработанного модуля для пользователя позволяют прогнозировать его коммерческий успех в будущем.

Следовательно, реализация программного модуля автоматизации курьерской службой на рынке является экономически эффективной и его разработку целесообразно осуществлять. 
