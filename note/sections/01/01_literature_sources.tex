\subsection{Аналитический обзор литературных источников}
\label{sec:analysis:literature}

Выбор литературных источников при подготовке к проектированию был обусловлен использованием определенных технологий, созданных для упрощения разработки веб-ориентированного программного обеспечения. В то же время, помимо печатных изданий, были использованы дополнительные ресурсы в сети Интернет.

Первым литературным источником, который использовался для получения информации по теме, стала книга автора Кайла Симпсона «You don’t know JS». Эта книга подробно описывает возможности и внутреннее устройство языка программирования JavaScript, а также тонкости использования предоставляемых им инструментов и работы с ним в целом. 

Вторым литературным источником, который использовался для получения информации по теме, стала книга автора Кайла Симпсона «Functional-Light JavaScript». Данный источник использовался с целью получения информации о функциональном подходе к программированию для применения их в разработке программного средства. Книга является вполне практико-ориентируемой - автор простым языком объясняет основные принципы функциональной парадигмы, не вдаваясь глубоко в теорию по этой теме, что позволяет быстро разобраться в материале и начать применять полученные знания на практике в разработке программного средства.

Третьим источником информации, который использовался для получения информации по теме, является документация технологии Webix - программного продукта, использовавшегося при разработке. Данная технология предоставляет большое количество уже спроектированных веб-компонентов со своей логикой работы и богатым инструментарием, что благоприятно сказывается на скорости разработки продуктов с использованием данной технологии.

Также стоит отметить сайт habrahabr.ru, который использовался для получения информации о REST - архитектурном стиле взаимодействия компонентов распределенного приложения.