\subsection{Разработка и спецификации функциональных требований}
\label{sec:analysis:specification}

Основной целью создаваемого программного средства является предоставление возможности создать собственные веб-компоненты на основании предоставляемых платформой или разработчиком, который интегрирует данное программное средство в свой продукт; скомпоновать из готовых “односложных” компонентов более комплексные, задав им при этом логику взаимодействия.
 
Пользователями данного программного средства могут быть как разработчики с целью более быстрой разработки интерфейса взаимодействия с пользователя с приложением, так и пользователи,которым разработчики, интегрирующие данное программное средство в свой продукт, пожелают предоставить возможность самим определить, как будет выглядеть для них дизайн веб-приложения.
Программное средство предоставляет интерфейс для взаимодействия с ним, позволяя гибко себя сконфигурировать. Однако, существует ряд требований к разработчику, интегрирующему данное программное средство в свой продукт:

\begin{itemize}
    \item Совместимость технологии, на которой осуществляется разработка продукта с технологией Webix (React, Angular, Ember.js);
    \item Разработчик должен задать место, куда будет интегрироваться программное средство;
    \item Разработчик должен предоставить способ получения конфигурационных объектов веб-компонентов, которые будут доступны пользователю изначально, в объект конфигурации (JSON, представляющий собой массив, содержащий объекты конфигураций компонентов или ссылку, по которой приложение сможет получить данный JSON).
\end{itemize}

Перечислим требования к программному средству:

\begin{itemize}
    \item Стабильная работа;
    \item Высокая скорость вычислений;
    \item Простота интеграции в готовые программные решения;
    \item Дружелюбный интерфейс;
    \item Возможность кастомизации;
    \item Возможность протестировать макет в песочнице;
    \item Возможность сохранять итоговый объект конфигурации скомпонованного пользователем макета;
    \item Обработка исключительных ситуаций и вывод информации о любых ошибках пользователю.
\end{itemize}

В качестве системы контроля версий был выбран Git, а в качестве GUI интерфейса для работы с Git используется программное средство GitKraken, так как обладает удобным, доступным интерфейсом, позволяющим легко ориентироваться в возможностях данного программного средства.
