\subsection{Выбор технологии разработки пользовательского интерфейса}
\label{sec:modeling:ui_technology}

Webix – это JavaScript и HTML5 UI библиотека, при помощи которой веб-разработчики могут создавать качественные кросс-браузерные и кросс-платформенные веб-приложения с отзывчивым дизайном. Библиотека доступна под двумя лицензиями: GNU GPLv3 и коммерческой.

Библиотека предлагает 70 готовых к использованию ui-виджетов, которые с легкостью настраиваются в соответствии с требованиями вашего проекта. Webix предлагает простую интеграцию с JQuery, Backbone, Angular и может работать с любой серверной платформой, например, PHP, .NET, Java и Ruby~\cite{what_is_webix}.

Особенности:

\begin{itemize}
    \item легкость освоения. Документация довольно подробна, и понять, как все устроено, несложно;
    \item интеграция с популярными фреймворками. Реализована интеграция с Backbone.js, AngularJS и jQuery. Последняя фича, например, позволяет создавать Webix-виджеты с использованием jQuery-синтаксиса;
    \item интеграция со сторонними виджетами. В этом пункте ограничимся списком: Mercury, Nicedit, Tinymce, CodeMirror, CKEditor, Raphael, D3, Sigma, JustGage, Google Maps, Nokia Maps, Yandex Maps, dhtmlxScheduler and dhtmlxGantt;
    \item размер — маленький, скорость — высокая. В сжатом виде .js-файл весит всего 128 КБ, и при этом все работает довольно-таки быстро (по словам разработчиков так и вовсе <<летает>>);
    \item поддержка тачскрина. Созданные виджеты одинаково хорошо себя чувствуют как на десктопах, так и на смартфонах/планшетах.
\end{itemize}

