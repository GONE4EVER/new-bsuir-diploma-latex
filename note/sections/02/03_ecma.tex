\subsection{Используемые стандарты}
\label{sec:modeling:ecma}

ECMAScript — это встраиваемый расширяемый не имеющий средств ввода-вывода язык программирования, используемый в качестве основы для построения других скриптовых языков. Стандартизирован международной организацией ECMA в спецификации ECMA-262. Расширения языка: JavaScript, JScript и ActionScript.

\subsubsection{Что такое ECMAScript}
\

Сначала немного истории. JavaScript создавался как скриптовый язык для Netscape. После чего он был отправлен в ECMA International для стандартизации (ECMA — это ассоциация, деятельность которой посвящена стандартизации информационных и коммуникационных технологий). Это привело к появлению нового языкового стандарта, известного как ECMAScript.

Последующие версии JavaScript уже были основаны на стандарте ECMAScript. Проще говоря, ECMAScript — стандарт, а JavaScript — самая популярная реализация этого стандарта.

\subsubsection{История версий}
\

ES — это просто сокращение для ECMAScript. Каждое издание ECMAScript получает аббревиатуру ES с последующим его номером. Всего существует 8 версий ECMAScript. ES1 была выпущена в июне 1997 года, ES2 — в июне 1998 года, ES3 — в декабре 1999 года, а версия ES4 — так и не была принята. Не будем углубляться в эти версии, так как они морально устарели, а рассмотрим только последние четыре.

ES5 был выпущен в декабре 2009 года, спустя 10 лет после выхода третьего издания. Среди изменений можно отметить:

\begin{itemize}
    \item поддержку строгого режима (strict mode);
    \item аксессоры getters и setters;
    \item возможность использовать зарезервированные слова в качестве ключей свойств и ставить запятые в конце массива;
    \item многострочные строковые литералы;
    \item новую функциональность в стандартной библиотеке;
    \item поддержку JSON.
\end{itemize}

Версия ES6/ES2015 вышла в июне 2015 года. Это также принесло некую путаницу в связи с названием пакета, ведь ES6 и ES2015 — это одно и то же. С выходом этого пакета обновлений комитет принял решение перейти к ежегодным обновлениям. Поэтому издание было переименовано в ES2015, чтобы отражать год релиза. Последующие версии также называются в соответствии с годом их выпуска. В этом обновлении были сделаны следующие изменения:

\begin{itemize}
    \item добавлено деструктурирующее присваивание;
    \item добавлены стрелочные функции;
    \item в шаблонных строках можно объявлять строки с помощью ` (обратных кавычек). Шаблонные строки могут быть многострочными, также могут интерполироваться;
    \item let и const — альтернативы var для объявления переменных. Добавлена <<временная мертвая зона>>;
    \item итератор и протокол итерации теперь определяют способ перебора любого объекта, а не только массивов. Symbol используется для присвоения итератора к любому объекту;
    \item добавлены функции-генераторы. Они используют yield для создания последовательности элементов. Функции-генераторы могут использовать yield* для делегирования в другую функцию генератора, кроме этого они могут возвращать объект генератора, который реализует оба протокола;
    \item добавлены промисы.
\end{itemize}

ES2016 (ES7) вышла в июне 2016 года. Среди изменений в этой версии ECMAScript можно отметить:

\begin{itemize}
    \item оператор возведения в степень **;
    \item метод Array.prototype.includes, который проверяет, содержится ли переданный аргумент в массиве.
\end{itemize}

Спустя еще год выходит версия ES2017 (ES8). Данный стандарт получил следующие изменения:

\begin{itemize}
    \item асинхронность теперь официально поддерживается (async/await);
    \item добавлена возможность ставить запятые в конце списка аргументов функций;
    \item добавлено два новых метода для работы со строками: padStart() и padEnd(). Метод padStart() подставляет дополнительные символы слева, перед началом строки. А padEnd(), в свою очередь, справа, после конца строки;
    \item добавлена функция Object.getOwnPropertyDescriptors(), которая возвращает массив с дескрипторами всех собственных свойств объекта;
    \item добавлено разделение памяти и объект Atomics.
    \item Что же касается ES.Next, то этот термин является динамическим и автоматически ссылается на новую версию ECMAScript. Стоит отметить, что каждая новая версия приносит новые функции для языка
\end{itemize}

Выводы:

\begin{itemize}
    \item ECMAscript выходит ежегодно;
    \item первые пакеты обновления назывались ES1, ES2, ES3, ES4, ES5;
    \item новые выпуски (начиная с 2015 года) получили название ES2015, ES2016, ES2017 (аббревиатура ES + год выпуска);
    \item ECMAScript является стандартом, а JavaScript — это самая популярная реализация этого стандарта.
\end{itemize}