\subsection{Выбор языка программирования}
\label{sec:modeling:language}

Для разработки программного средства создания веб-приложений с помощью 
готовых графических компонентов был выбран язык программирования JavaScript и технология Webix - библиотека для написания UI приложения.

Гипертекстовая информационная система состоит из множества информационных узлов, множества гипертекстовых связей, определенных на этих узлах и инструментах манипулирования узлами и связями. Технология World Wide Web - это технология ведения гипертекстовых распределенных систем в Internet , и, следовательно, она должна соответствовать общему определению таких систем. Это означает, что все перечисленные выше компоненты гипертекстовой системы должны быть и в Web.

Web, как гипертекстовую систему, можно рассматривать с двух точек зрения. Во-первых, как совокупность отображаемых страниц, связанных гипертекстовыми переходами (ссылками - контейнер ANCHOR). Во-вторых, как множество элементарных информационных объектов, составляющих отображаемые страницы (текст, графика, мобильный код и т.п.). В последнем случае множество гипертекстовых переходов страницы - это такой же информационный фрагмент, как и встроенная в текст картинка.

При втором подходе гипертекстовая сеть определяется на множестве элементарных информационных объектов самими HTML-страницами, которые и играют роль гипертекстовых связей. Этот подход более продуктивен с точки зрения построения отображаемых страниц <<на лету>> из готовых компонентов.

При генерации страниц в Web возникает дилемма, связанная с архитектурой <<клиент-сервер>>. Страницы можно генерировать как на стороне клиента, так и на стороне сервера. В 1995 году специалисты компании Netscape создали механизм управления страницами на клиентской стороне, разработав язык программирования JavaScript.

Таким образом, JavaScript - это язык управления сценариями просмотра гипертекстовых страниц Web на стороне клиента. Если быть более точным, то JavaScript - это не только язык программирования на стороне клиента. Liveware, прародитель JavaScript, является средством подстановок на стороне сервера Netscape. Однако наибольшую популярность JavaScript обеспечило программирование на стороне клиента.

Основная идея JavaScript состоит в возможности изменения значений атрибутов HTML-контейнеров и свойств среды отображения в процессе просмотра HTML-страницы пользователем. При этом перезагрузки страницы не происходит.

На практике это выражается в том, что можно, например, изменить цвет фона страницы или интегрированную в документ картинку, открыть новое окно или выдать предупреждение.

Название <<JavaScript>> является собственностью Netscape. Реализация языка, осуществленная разработчиками Microsoft, официально называется Jscript. Версии JScript совместимы (если быть совсем точным, то не до конца) с соответствующими версиями JavaScript, т.е. JavaScript является подмножеством языка JScript.

JavaScript стандартизован ECMA (European\-Computer\-Manufacturers\-Association - Ассоциация европейских производителей компьютеров). Соответствующие стандарты носят названия ECMA-262 и ISO-16262. Этими стандартами определяется язык ECMAScript, который примерно эквивалентен JavaScript 1.1. Отметим, что не все реализации JavaScript на сегодня полностью соответствуют стандарту ECMA. В рамках данного курса мы во всех случаях будем использовать название JavaScript.

JavaScript - объектно-ориентированный скриптовый язык программирования. Является диалектом языка ECMAScript.

JavaScript обычно используется как встраиваемый язык для программного доступа к объектам приложений. Наиболее широкое применение находит в браузерах как язык сценариев для придания интерактивности веб-страницам.

Основные архитектурные черты: динамическая типизация, слабая типизация, автоматическое управление памятью, прототипное программирование, функции как объекты первого класса.

На JavaScript оказали влияние многие языки, при разработке была цель сделать язык похожим на Java, но при этом лёгким для использования непрограммистами. Языком JavaScript не владеет какая-либо компания или организация, что отличает его от ряда языков программирования, используемых в веб-разработке.

Название <<JavaScript>> является зарегистрированным товарным знаком компании Oracle Corporation~\cite{wiki_js}.

\subsubsection{Возможности языка программирования JavaScript}
\

JavaScript обладает рядом свойств объектно-ориентированного языка, но реализованное в языке прототипирование обусловливает отличия в работе с объектами по сравнению с традиционными объектно-ориентированными языками. Кроме того, JavaScript имеет ряд свойств, присущих функциональным языкам - функции как объекты первого класса, объекты как списки, карринг, анонимные функции, замыкания - что придаёт языку дополнительную гибкость.

Несмотря на схожий с С синтаксис, JavaScript по сравнению с языком Си имеет коренные отличия:

\begin{itemize}
  \item объекты, с возможностью интроспекции;
  \item функции как объекты первого класса;
  \item автоматическое приведение типов;
  \item автоматическая сборка мусора;
  \item анонимные функции.
\end{itemize}

В языке отсутствуют такие полезные вещи, как:

\begin{itemize}
  \item модульная система: JavaScript не предоставляет возможности управлять зависимостями и изоляцией областей видимости;
  \item стандартная библиотека: в частности, отсутствует интерфейс программирования приложений по работе с файловой системой, управлению потоками ввода/вывода, базовых типов для бинарных данных;
  \item стандартные интерфейсы к веб-серверам и базам данных;
  \item система управления пакетами, которая бы отслеживала зависимости и автоматически устанавливала их.
\end{itemize}

Синтаксис языка JavaScript во многом напоминает синтаксис Си и Java, семантически же язык гораздо ближе к Self, Smalltalk или даже Лиспу.

В JavaScript:

\begin{itemize}
  \item все идентификаторы регистрозависимы;
  \item в названиях переменных можно использовать буквы, подчёркивание, символ доллара, арабские цифры;
  \item названия переменных не могут начинаться с цифры;
  \item для оформления однострочных комментариев используются //, многострочные и \item внутристрочные комментарии начинаются с /* и заканчиваются */.
\end{itemize}

Для создания механизма управления страницами на клиентской стороне было предложено использовать объектную модель документа. Суть модели в том, что каждый HTML-контейнер — это объект, который характеризуется тройкой:

\begin{itemize}
  \item свойства;
  \item методы;
  \item события.
\end{itemize}

Объектную модель можно представить как способ связи между страницами и браузером. Объектная модель — это представление объектов, методов, свойств и событий, которые присутствуют и происходят в программном обеспечении браузера, в виде, удобном для работы с ними кода HTML и исходного текста сценария на странице. Мы можем с ее помощью сообщать наши пожелания браузеру и далее — посетителю страницы. Браузер выполнит наши команды и соответственно изменит страницу на экране.

Объекты с одинаковым набором свойств, методов и событий объединяются в классы однотипных объектов. Классы — это описания возможных объектов. Сами объекты появляются только после загрузки документа браузером или как результат работы программы. Об этом нужно всегда помнить, чтобы не обратиться к объекту, которого нет.