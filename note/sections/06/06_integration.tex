\subsection{Интеграция программного средства}
\label{sec:manual:integration}

Программное средство создания веб-приложений с помощью готовых графических компонентов позиционируется как интегрируемый сервис. В конечном счете оно представляет собой набор всего лишь трех основных компонентов: палитры компонентов, грида и инспектора свойств. 

Несмотря на всю гибкость предоставляемых возможностей, есть определенные требования к тому, как данное программное средство будет интегрироваться:
\begin{itemize}
	\item при инициализации можно создать сколько угодно палитр компонентов и инспекторов свойств, но каждый из них может быть связан только с одним гридом;
	\item при инициализации грид компоненту необходимо указать уникальный идентификатор, а в палитре компонентов и инспекторе свойств указать этот индентификатор в качестве значения поля с ключом <<gridId>>;
	\item при инициализации палитры компонентов, необходимо указать свойство с ключом <<widgets>>, значению которого будет соответствовать адрес узла сервера, который предоставит все необходимые данные в нужном формате;
	\item загружаемые данные должны соответствовать строго определенному формату;
	\item необходимо указать способ загрузки пресетов и компонентов: можно предоставить непосредственно набор данных или загружать их при помощи самих компонентов. 
\end{itemize}

Спецификация формата данных выгляит следующим образом:
\begin{lstlisting}
{
  config: { 
    // config content
  },
  element: 'webix_view_name',
  icon: 'icon_name'
}
\end{lstlisting}

Краткое текстовое пояснение к спецификации определения формата данных программного средства создания веб-приложений, в частности палитры компонентов, звучит следующим образом:
\begin{itemize}
	\item поле <<config>> должно содержать JavaScript-объект -- конфигурацию компонента в соответствии с документацией библиотеки Webix;
	\item значению поля <<element>> должно соответствовать строковое значение, являющееся названием компонента библиотеки Webix;
	\item значению поля <<icon>> должно соответствовать строковое значение, являющееся названием иконки в библиотеке font-awesome.
\end{itemize}

Что же до пресетов, то, ввиду того, что являются отдельной категорией данных, определение способа их загрузки остается за разработчиком, интегрирующим данное программное средство в свой продукт.

Конфиг для инициализации программного средства в другой продукт в коненом счете будет иметь вид, указанный в участе кода ниже.

\begin{lstlisting}
webix.ui({
	container: 'root',
	cols: [
	  {
	    view: 'AppOrchid-Component-Pallet',
	    widgets: url,
	    gridId: 'Grid1',
	    width: 170
	  },
	  {
	    id: 'Grid1',
	    gravity: 2,
	    view: 'AppOrchid-Components-Layout'
	  },
	  {
	    view: 'AppOrchid-Components-Settings',
	    gridId: 'Grid1',
	    gravity: 1
	  }
	]
});
\end{lstlisting}

Вместо константы url, показанной выше, следует поместить ссылку на адрес узла сервера, который предоставит все необходимые данные в нужном формате.

Свойство <<gravity>> нужно для задания пропорций, с которыми будут отображаться компоненты -- например, в вышеуказанном коде палитра компонентов имеет фиксированную ширину 170 пикселей, а остальные два компонента делят оставшуюся ширину контейнера в соотношении 2 к 1.

В проекте используются нововведения и синтаксис ES6~\cite{ecma_additional}, ввиду чего в проекте, в который будет интегрироваться данное программное средство, обязан использоваться Babel~\cite{babel}, иначе проект просто не будет работать. 

Ниже будет представлен полный список использованных при разработке проекта зависимостей. 

\begin{lstlisting}
{
	"dependencies": {
		"body-parser": "^1.18.2",
		"copy-webpack-plugin": "^4.5.1",
		"css-loader": "^0.28.11",
		"express": "^4.16.3",
		"express-fileupload": "^0.4.0",
		"file-loader": "^1.1.11",
		"html-webpack-plugin": "^3.1.0",
		"less": "^3.8.1",
		"less-loader": "^4.1.0",
		"style-loader": "^0.20.3",
		"ts-loader": "^4.5.0",
		"tslint": "^5.11.0",
		"typescript": "^3.0.1",
		"url-loader": "^1.0.1",
		"webpack-dev-server": "^3.1.5"
	},
	"devDependencies": {
		"babel-core": "^6.26.0",
		"babel-loader": "^7.1.4",
		"babel-plugin-module-resolver": "^3.1.1",
		"babel-plugin-transform-object-rest-spread": "^6.26.0",
		"babel-polyfill": "^6.26.0",
		"babel-preset-env": "^1.6.1",
		"eslint": "^4.19.1",
		"eslint-plugin-import": "^2.9.0",
		"eslint-xbsoftware": "git+https://bitbucket.org/xbsltd/eslint.git",
		"extract-text-webpack-plugin": "^4.0.0-beta.0",
		"resolve-url-loader": "^2.3.0",
		"webpack": "^4.16.5",
		"webpack-cli": "^2.0.13"
	}
}
\end{lstlisting}

Резюмируя всё вышесказанное, можно сделать вывод, что при разработке приложения применялись приемы, благоприятно воздействующие на UX, а также обеспечивающие более легкую разработку, возможное будущее расширение и поддержку приложения.