\subsection{Интеграция программного средства}
\label{sec:manual:integration}

Данное программное средство позиционируется как интегрируемый сервис. В конечном счете оно представляет собой набор всего лишь трех основных компонентов: палитры компонентов, грида и инспектора свойств. 

Несмотря на всю гибкость предоставляемых возможностей, есть определенные требования к тому, как данное программное средство будет интегрироваться:
\begin{itemize}
	\item при инициализации можно создать сколько угодно палитр компонентов и инспекторов свойств, но каждый из них может быть связан только с одним гридом;
	\item при инициализации грид компоненту необходимо указать уникальный идентификатор, а в палитре компонентов и инспекторе свойств указать этот индентификатор в качестве значения поля с ключом <<gridId>>;
	\item при инициализации палитры компонентов, необходимо указать свойство с ключом <<widgets>>, значению которого будет соответствовать адрес узла сервера, который предоставит все необходимые данные в нужном формате;
	\item загружаемые данные должны соответствовать строго определенному формату;
	\item необходимо указать способ загрузки пресетов и компонентов: можно предоставить непосредственно набор данных или загружать их при помощи самих компонентов. 
\end{itemize}

Спецификация формата данных выгляит следующим образом:
\begin{lstlisting}
{
  config: { 
    // config content
  },
  element: 'webix_view_name',
  icon: 'area-chart'
}
\end{lstlisting}

Краткое текстовое пояснение к спецификации определения формата данных программного средства создания веб-приложений, в частности палитры компонентов, звучит следующим образом:
\begin{itemize}
	\item поле <<config>> должно содержать JavaScript-объект - конфигурацию компонента в соответствии с документацией библиотеки Webix;
	\item значению поля <<element>> должно соответствовать строковое значение, являющееся названием компонента библиотеки Webix;
	\item значению поля <<icon>> должно соответствовать строковое значение, являющееся названием иконки в библиотеке font-awesome.
\end{itemize}

Что же до пресетов, то, ввиду того, что являются отдельной категорией данных, определение способа их загрузки остается за разработчиком, интегрирующим данное программное средство в свой продукт.

Конфиг для инициализации программного средства в другой продукт в коненом счете будет иметь вид, указанный в участе кода ниже.

\begin{lstlisting}
webix.ui({
	container: 'root',
	cols: [
	  {
	    view: 'AppOrchid-Component-Pallet',
	    widgets: url,
	    gridId: 'Grid1',
	    width: 170
	  },
	  {
	    id: 'Grid1',
	    gravity: 2,
	    view: 'AppOrchid-Components-Layout'
	  },
	  {
	    view: 'AppOrchid-Components-Settings',
	    gridId: 'Grid1',
	    gravity: 1
	  }
	]
});
\end{lstlisting}

Вместо константы url, показанной выше, следует поместить ссылку на адрес узла сервера, который предоставит все необходимые данные в нужном формате.

Свойство <<gravity>> нужно для задания пропорций, с которыми будут отображаться компоненты - например, в вышеуказанном коде палитра компонентов имеет фиксированную ширину 170 пикселей, а остальные два компонента делят оставшуюся ширину контейнера в соотношении 2 к 1.

Резюмируя всё вышесказанное, можно сделать вывод, что при разработке приложения применялись приемы, благоприятно воздействующие на UX, а также обеспечивающие более легкую разработку, возможное будущее расширение и поддержку приложения.