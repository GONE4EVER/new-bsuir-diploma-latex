\section{Тестирование и проверка работоспособности ПС}
\label{sec:testing}

В данном разделе проведем динамическое ручное тестирование. В таблице~\ref{table:testing:test_cases} приведен список тестовых случаев, относящихся к позитивному тестированию.

Тестирование производилось при версиях зависимостей, указанных ниже. Ввиду того, что некоторые зависимости могли устареть, а некоторые их функции -- перестать поддерживаться разработчиками пакета, стабильная работа на более поздних версиях пакетов не гарантируется.

\begin{lstlisting}
{
	"dependencies": {
      "body-parser": "^1.18.2",
      "copy-webpack-plugin": "^4.5.1",
      "css-loader": "^0.28.11",
      "express": "^4.16.3",
      "express-fileupload": "^0.4.0",
      "file-loader": "^1.1.11",
      "html-webpack-plugin": "^3.1.0",
      "less": "^3.8.1",
      "less-loader": "^4.1.0",
      "style-loader": "^0.20.3",
      "ts-loader": "^4.5.0",
      "tslint": "^5.11.0",
      "typescript": "^3.0.1",
      "url-loader": "^1.0.1",
      "webpack-dev-server": "^3.1.5"
	},
	"devDependencies": {
      "babel-core": "^6.26.0",
      "babel-loader": "^7.1.4",
      "babel-plugin-module-resolver": "^3.1.1",
      "babel-plugin-transform-object-rest-spread": "^6.26.0",
      "babel-polyfill": "^6.26.0",
      "babel-preset-env": "^1.6.1",
      "eslint": "^4.19.1",
      "eslint-plugin-import": "^2.9.0",
      "extract-text-webpack-plugin": "^4.0.0-beta.0",
      "resolve-url-loader": "^2.3.0",
      "webpack": "^4.16.5",
      "webpack-cli": "^2.0.13"
	}
}
\end{lstlisting}

% Зачем: свой счетчик для нумерации тестов.
\newcounter{testnumber}
\newcommand\testnumber{\stepcounter{testnumber}\arabic{testnumber}}

% Переключаем команды нумерации для шагов тестов. В конце файла вернем всё как было.
\renewcommand{\labelenumi}{\arabic{enumi})}
\renewcommand{\labelenumii}{\asbuk{enumii})}

\begin{longtable}{|>{\centering}m{0.065\textwidth}
  |p{0.36\textwidth}
  |p{0.307\textwidth}
  |>{\centering\arraybackslash}m{0.163\textwidth}|} 
\caption{Тестовые случаи разрабатываемого программного средства}
\label{table:testing:test_cases}\\

\hline
\begin{minipage}{1\linewidth}
  \centering №
\end{minipage} & 
\begin{minipage}{1\linewidth}
	\centering Описание тестового случая
\end{minipage} & 
\begin{minipage}{1\linewidth}
	\centering Ожидаемый результат
\end{minipage} & 
\centering\arraybackslash Полученный результат \endfirsthead

\caption*{Продолжение таблицы \ref{table:testing:test_cases}}\\\hline
\centering № & \centering Описание тестового случая & \centering Ожидаемый результат & \centering\arraybackslash Полученный результат \\\hline \endhead

% 1
\hline
\testnumber &
\begin{minipage}[t]{1\linewidth}
	\textbf{Перетягивание обычного компонента на грид}.
  \begin{enumerate}
		\item зажать левую кнопку мыши на компоненте, не относящемся к типу <<layout>>;
		\item перенянуть на грид;
		\item отпустить левую кнопку мыши.
	\end{enumerate}
\end{minipage} &
Компонент не отрисовывается на гриде. Отображается сообщение об ошибке, что компонент не относится к типу <<layout>>. & Пройдено \\
\hline

% 2
\testnumber &
\begin{minipage}[t]{1\linewidth}
	\textbf{Выделение лэйаута}.
  \begin{enumerate}
    \item перетянуть <<Horizontal layout>> на грид;
    \item левой кнопкой мыши нажать на созданный <<Horizontal layout>>,.
  \end{enumerate}
\end{minipage} &
Границы компонента окрашиваются прерывистой зеленой линией, и на границе появляются точки для изменения размеров компонента. & Пройдено \\
\hline

% 3
\testnumber &
\begin{minipage}[t]{1\linewidth}
	\textbf{Добавление компонента в лэйаут}.
  \begin{enumerate}
    \item перетянуть <<Horizontal layout>> на грид;
    \item перетянуть <<Combo>> в созданный <<Horizontal layout>>.
  \end{enumerate}
\end{minipage} &
Компонент <<Combo>> будет отрисован внутри созданного <<Horizontal layout>>. & Пройдено \\
\hline

% 4
\testnumber &
\begin{minipage}[t]{1\linewidth}
	\textbf{Выделение компонента внутри лэйаута}.
  \begin{enumerate}
    \item перетянуть <<Horizontal layout>> на грид;
    \item перетянуть <<Combo>> в созданный <<Horizontal layout>>;
    \item левой кнопкой мыши нажать на созданный <<Combo>>.
  \end{enumerate}
\end{minipage} &
Границы компонента окрашиваются прерывистой зеленой линией, а на границе не появляются точки для изменения размеров компонента. & Пройдено \\

% 5
\testnumber &
\begin{minipage}[t]{1\linewidth}
	\textbf{Редактирование компонента}.
  \begin{enumerate}
    \item перетянуть <<Horizontal layout>> на грид;
    \item перетянуть <<Combo>> в созданный <<Horizontal layout>>;
    \item левой кнопкой мыши нажать на созданный <<Combo>>;
    \item открыть вкладку <<STYLE>> в инспекторе свойств;
    \item левой кнопкой мыши нажать на  на значение поля <<width>>;
    \item используя только цифры, ввести данные о новых размерах;
    \item левой кнопкой мыши нажать на кнопку <<Enter>>.
  \end{enumerate}
\end{minipage} &
Компонент перерисуется в соответствии с введенными данными. & Пройдено \\

% 6
\testnumber &
\begin{minipage}[t]{1\linewidth}
	\textbf{Редактирование лэйаута}.
  \begin{enumerate}
    \item перетянуть <<Horizontal layout>> на грид;
    \item перетянуть <<Combo>> в созданный <<Horizontal layout>>;
    \item выделить созданный <<Horizontal layout>>;
    \item открыть вкладку <<STYLE>> в инспекторе свойств;
    \item левой кнопкой мыши нажать на значение поля <<width>> или <<height>>;
    \item используя только цифры, ввести данные о новых размерах;
    \item левой кнопкой мыши нажать кнопку <<Enter>>.
  \end{enumerate}
\end{minipage} &
Лэйаут и его дочерние элементы перерисовываются в соответствии с введенными данными. & Пройдено \\
\hline

% 7
\testnumber &
\begin{minipage}[t]{1\linewidth}
	\textbf{Добавление компонента в конец непустого лэйаута}.
  \begin{enumerate}
    \item перетянуть <<Horizontal layout>> на грид;
    \item перетянуть <<Combo>> в созданный <<Horizontal layout>>;
    \item перетянуть <<Datatable>> на созданный <<Horizontal layout>>.
  \end{enumerate}
\end{minipage} &
Компонент <<Datatable>> будет отрисован крайним справа внутри созданного <<Horizontal layout>>. & Пройдено \\

% 8
\testnumber &
\begin{minipage}[t]{1\linewidth}
	\textbf{Добавление компонента перед определенным компонентов внутри непустого лэйаута}.
  \begin{enumerate}
    \item перетянуть <<Horizontal layout>> на грид;
    \item перетянуть <<Combo>> в созданный <<Horizontal layout>>;
    \item перетянуть <<Datatable>> на созданный <<Combo>>.
  \end{enumerate}
\end{minipage} &
Компонент <<Datatable>> будет отрисован перед компонентом <<Combo>> внутри созданного <<Horizontal layout>>. & Пройдено \\
\hline

% 9
\testnumber &
\begin{minipage}[t]{1\linewidth}
	\textbf{Удаление компонента внутри лэйаута с одним дочерним компонентом}.
  \begin{enumerate}
    \item перетянуть <<Horizontal layout>> на грид;
    \item перетянуть <<Combo>> в созданный <<Horizontal layout>>;
    \item левой кнопкой мыши нажать на созданный <<Combo>>;
    \item открыть вкладку <<STYLE>> в инспекторе свойств;
    \item левой кнопкой мыши нажать на кнопку <<DELETE COMPONENT>>.
  \end{enumerate}
\end{minipage} &
Компонент <<Combo>> будет удален внутри созданного <<Horizontal layout>> Сам лэйаут останется на гриде. & Пройдено \\

% 10
\testnumber &
\begin{minipage}[t]{1\linewidth}
	\textbf{Удаление непустого лэйаута}.
  \begin{enumerate}
    \item перетянуть <<Horizontal layout>> на грид;
    \item перетянуть <<Combo>> в созданный <<Horizontal layout>>;
    \item левой кнопкой мыши нажать на созданный <<Horizontal layout>>;
    \item открыть вкладку <<STYLE>> в инспекторе свойств;
    \item левой кнопкой мыши нажать на кнопку <<DELETE COMPONENT>>.
  \end{enumerate}
\end{minipage} &
Компонент <<Horizontal layout>> будет удален с грида вместе с дочерним элементом <<Combo>>. & Пройдено \\
\hline

% 11
\testnumber &
\begin{minipage}[t]{1\linewidth}
	\textbf{Многоуровневая компоновка}.
  \begin{enumerate}
		\item перетянуть <<Horizontal layout>> на грид;
		\item перетянуть <<Datatable>> в созданный <<Horizontal layout>>;
		\item перенянуть <<Vertical layout>> в созданный <<Horizontal layout>>;
		\item перенянуть <<Combo>> в созданный <<Vertical layout>>;
		\item перенянуть <<Datepicker>> в созданный <<Vertical layout>>.
	\end{enumerate}
\end{minipage} &
Отображается компонент, состоящий из двух колонок: в первой <<Datatable>>, во второй в строки расположены: <<Combo>> и <<Datepicker>>.& Пройдено \\

% 12
\testnumber &
\begin{minipage}[t]{1\linewidth}
	\textbf{Удаление первого компонента среди дочерних компонентов лэйаута}.
  \begin{enumerate}
		\item перетянуть <<Horizontal layout>> на грид;
		\item перетянуть <<Datatable>> в созданный <<Horizontal layout>>;
		\item перенянуть <<Vertical layout>> в созданный <<Horizontal layout>>;
		\item перенянуть <<Combo>> в созданный <<Vertical layout>>;
		\item перенянуть <<Datepicker>> в созданный <<Vertical layout>>;
		\item левой кнопкой мыши нажать на <<Datatable>>;
		\item открыть вкладку <<STYLE>> в инспекторе свойств;
    \item левой кнопкой мыши нажать на кнопку <<DELETE COMPONENT>>.
	\end{enumerate}
\end{minipage} &
Родительский <Horizontal layout>> перерисовывается, компонент <<Datatable>> удалятся,  и <<Vertical layout>> с компонентами <<Combo>> и <<Datepicker>> занимает его место.& Пройдено \\

% 13
\testnumber &
\begin{minipage}[t]{1\linewidth}
	\textbf{Появление тени от лэйаута на гриде}.
  \begin{enumerate}
		\item перетянуть <<Horizontal layout>> на грид;
		\item перетянуть <<Datepicker>> на созданный <<Horizontal layout>>;
		\item зажать левую кнопку мыши на созданном <<Horizontal layout>>;
		\item перетянуть лэйаут с зажатой левой кнопкой мыши в другое место грида.
	\end{enumerate}
\end{minipage} &
Сзади компонента появляется <<тень>>.& Пройдено \\
\hline

% 14
\testnumber &
\begin{minipage}[t]{1\linewidth}
	\textbf{Магнитная сетка}.
  \begin{enumerate}
		\item перетянуть <<Horizontal layout>> на грид;
		\item перетянуть <<Datepicker>> на созданный <<Horizontal layout>>;
		\item зажать левую кнопку мыши на созданном <<Horizontal layout>>;
		\item медленно перегивать лэйаут с зажатой левой кнопкой мыши;
		\item отпустить левую кнопку мыши.
	\end{enumerate}
\end{minipage} &
Сзади компонента появляется <<тень>>, двигающаяся по гриду с шагом, равным размеру ребра клетки грида. После отпускания левой кнопки мыши компонент помещен туда, куда падала его тень.& Пройдено \\

% 15
\testnumber &
\begin{minipage}[t]{1\linewidth}
	\textbf{Изменение позиции созданного непустого лэйаута}.
  \begin{enumerate}
		\item перетянуть <<Horizontal layout>> на грид;
		\item перетянуть <<Datepicker>> на созданный <<Horizontal layout>>;
		\item зажать левую кнопку мыши на созданном <<Horizontal layout>>;
		\item перетянуть лэйаут в другое место грида;
		\item отпустить левую кнопку мыши.
	\end{enumerate}
\end{minipage} &
Созданный на гриде <<Horizontal layout>> передвинут вмете с содержимым с начального положения.& Пройдено \\
\hline

% 16
\testnumber &
\begin{minipage}[t]{1\linewidth}
	\textbf{Редактирование уникальных свойств компонентов}.
  \begin{enumerate}
		\item перетянуть <<Horizontal layout>> на грид;
		\item перетянуть <<Leaf>> на созданный <<Horizontal layout>>;
		\item левой кнопкой мыши нажать на созданный компонент <<Leaf>>;
		\item открыть вкладку <<STYLE>> в инспекторе свойств;
    \item левой кнопкой мыши нажать на значение свойства <<color>>;
    \item левой кнопкой мыши нажать на нужный цвет в выпадающем колопикере.
	\end{enumerate}
\end{minipage} &
В созданном компоненте <<Leaf>> изменится цвет шаблона.& Пройдено \\

% 17
\testnumber &
\begin{minipage}[t]{1\linewidth}
	\textbf{Редактирование уникальных свойств компонентов}.
  \begin{enumerate}
		\item перетянуть <<Horizontal layout>> на грид;
		\item перетянуть <<Leaf>> на созданный <<Horizontal layout>>;
		\item левой кнопкой мыши нажать на созданный компонент <<Chart>>;
		\item открыть вкладку <<STYLE>> в инспекторе свойств;
    \item левой кнопкой мыши нажать на значение свойства <<type>>;
    \item ввести строку <<pie>>;
    \item левой кнопкой мыши нажать кнопку <<Enter>>;
    \item ввести строку <<line>>;
    \item левой кнопкой мыши нажать кнопку <<Enter>>.
	\end{enumerate}
\end{minipage} &
В созданном компоненте <<Chart>> меняется тип графика сначала на <<pie>>, затем на <<line>>. Проверка соответсвия вида графика осуществляется согласно документации Webix.& Пройдено \\

% 18
\testnumber &
\begin{minipage}[t]{1\linewidth}
	\textbf{Отображение дополнительных свойств компонента <<Datatable>>}.
  \begin{enumerate}
		\item перетянуть <<Horizontal layout>> на грид;
		\item перетянуть <<Datatable>> на созданный <<Horizontal layout>>;
		\item левой кнопкой мыши нажать на созданный компонент <<Datatable>>;
		\item открыть вкладку <<STYLE>> в инспекторе свойств.
	\end{enumerate}
\end{minipage} &
Во вкладке со свойствами появляется дополнительная вкладка для колонок компонента <<Datatable>>.& Пройдено \\
\hline

% 19
\testnumber &
\begin{minipage}[t]{1\linewidth}
	\textbf{Отображение кода выделенного компонента при выборе вкладки <<CODE>>}.
  \begin{enumerate}
		\item перетянуть <<Horizontal layout>> на грид;
		\item открыть вкладку <<CODE>> в инспекторе свойств.
	\end{enumerate}
\end{minipage} &
Содержимое инспектора инструментов перерисуется, появится содержимое вкладки <<CODE>>: textarea с JavaScript-кодом объекта компонента <<Horizontal layout>>.& Пройдено \\

% 20
\testnumber &
\begin{minipage}[t]{1\linewidth}
	\textbf{Переключение вкладок палитры компонентов}.
  \begin{enumerate}
		\item нажать на троеточие рядом со вкладкой <<All>> палитры компонентов;
		\item в появившемся попапе нажать на <<Components>>;
		\item нажать на троеточие рядом со вкладкой <<Components>> палитры компонентов;
		\item в появившемся попапе левой кнопкой мыши нажать на <<Layouts>>;
		\item нажать на троеточие рядом со вкладкой <<Layouts>> палитры компонентов;
		\item в появившемся попапе левой кнопкой мыши нажать на <<Custom templates>>.
	\end{enumerate}
\end{minipage} &
Происходит перерисовка содержимого списка палитры компонентов в соответствии с выбранной вкладкой.& Пройдено \\

% 21
\testnumber &
\begin{minipage}[t]{1\linewidth}
	\textbf{Сохранение нового пресета}.
  \begin{enumerate}
		\item перетянуть <<Horizontal layout>> на грид;
		\item перетянуть <<Datatable>> на созданный <<Horizontal layout>>;
		\item нажать на троеточие рядом со вкладкой <<All>> палитры компонентов;
		\item в появившемся попапе нажать на <<Presets>>;
		\item левой кнопкой мыши нажать на кнопку <<SAVE CURRENT CONFIGURATION>>;
		\item ввести <<123>> в появившемся попапе;
		\item левой кнопкой мыши нажать на кнопку <<ADD>>.
	\end{enumerate}
\end{minipage} &
В список пресетов в самый низ добавляется запись с именем <<123>>.& Пройдено \\

% 22
\testnumber &
\begin{minipage}[t]{1\linewidth}
	\textbf{Создание нового пресета с пустым именем}.
  \begin{enumerate}
		\item перетянуть <<Horizontal layout>> на грид;
		\item перетянуть <<Datatable>> на созданный <<Horizontal layout>>;
		\item левой кнопкой мыши нажать на троеточие рядом со вкладкой <<All>> палитры компонентов;
		\item в появившемся попапе нажать на <<Presets>>;
		\item левой кнопкой мыши нажать на кнопку <<SAVE CURRENT CONFIGURATION>>;
		\item левой кнопкой мыши нажать на кнопку <<ADD>>.
	\end{enumerate}
\end{minipage} &
В правом верхнем углу экрана появится сообщение об ошибке с текстом <<Preset name can`t be empty>>.& Пройдено \\

% 23
\testnumber &
\begin{minipage}[t]{1\linewidth}
	\textbf{Создание нового пресета с именем уже существующего пресета}.
  \begin{enumerate}
		\item перетянуть <<Horizontal layout>> на грид;
		\item перетянуть <<Datatable>> на созданный <<Horizontal layout>>;
		\item нажать на троеточие рядом со вкладкой <<All>> палитры компонентов;
		\item в появившемся попапе нажать на <<Presets>>;
		\item левой кнопкой мыши нажать на кнопку <<SAVE CURRENT CONFIGURATION>>;
		\item ввести <<123>> в появившемся попапе;
		\item левой кнопкой мыши нажать на кнопку <<ADD>>;
		\item левой кнопкой мыши нажать на кнопку <<SAVE CURRENT CONFIGURATION>>;
		\item повторить шаги 7 и 8.
	\end{enumerate}
\end{minipage} &
В правом верхнем углу экрана появляется сообщение об ошибке с текстом <<Item with this name already exists>>, а в список пресетов не добавляется новый пресет с введенным именем.& Пройдено \\

% 24
\testnumber &
\begin{minipage}[t]{1\linewidth}
	\textbf{Загрузка пресета}.
  \begin{enumerate}
		\item перетянуть <<Horizontal layout>> на грид;
		\item перетянуть <<Datatable>> на созданный <<Horizontal layout>>;
		\item нажать на троеточие рядом со вкладкой <<All>> палитры компонентов;
		\item в появившемся попапе нажать на <<Presets>>;
		\item левой кнопкой мыши нажать на кнопку <<SAVE CURRENT CONFIGURATION>>;
		\item ввести <<123>> в появившемся попапе;
		\item левой кнопкой мыши нажать на кнопку <<ADD>>;
		\item левой кнопкой мыши дважды нажать на появившуся в списке надпись с текстом <<123>>.
	\end{enumerate}
\end{minipage} &
Грид полностью очищается от содержимого, после чего на него добавляется <<Horizontal layout>> с компонентом <<Datatable>> внутри.& Пройдено \\

% 25
\testnumber &
\begin{minipage}[t]{1\linewidth}
	\textbf{Очистка содержимого грида}.
  \begin{enumerate}
		\item перетянуть <<Horizontal layout>> на грид;
		\item перетянуть <<Datatable>> на созданный <<Horizontal layout>>;
		\item перетянуть <<Vertical layout>> на грид;
		\item перетянуть <<Combo>> на созданный <<Vertical layout>>;
		\item нажать на троеточие рядом со вкладкой <<All>> палитры компонентов;
		\item в появившемся попапе нажать на <<Presets>>;
		\item левой кнопкой мыши нажать на кнопку <<CLEAR LAYOUT>>. 
	\end{enumerate}
\end{minipage} &
Грид полностью очищается от содержимого.& Пройдено \\
\hline

\end{longtable}


% Зачем: возвращаем нумерацию перечислений как надо по стандарту.
\renewcommand{\labelenumi}{\asbuk{enumi})}
\renewcommand{\labelenumii}{\arabic{enumii})}

По итогам тестирования можно сделать вывод, что программное средство было разработано устойчивым к разного рода ошибкам как внутренней логики работы различных модулей и их взаимодействия, так логики взаимодействия с пользователем.