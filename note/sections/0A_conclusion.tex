\sectioncentered*{Заключение}
\addcontentsline{toc}{section}{Заключение}

Предметной областью данного дипломного проекта является информатизация некоторых задач участников учебного процесса: студентов и преподавателей. Был проведен поиск существующих программные средств этого рода, по его результатам был сделан вывод о несуществовании полных аналогов. Студенты и преподаватели адаптируют различные средства для использования в профессиональной деятельности; были выявлены недостатки данного подхода. Было предложено программное средство, которое должно унифицировать используемые средства и подходы в процессе обучения касательно задач управления временем, задачами и коммуникацией. 

На основании проведенного анализа предметной области были выдвинуты требования к программному средству. В качестве технологий разработки были выбраны наиболее современные существующие на данный момент средства, широко применяемые в индустрии. Спроектированное программное средство было успешно протестировано на соответствие спецификации функциональных требований. Уже исходя только из анализа предметной области и факта несуществования полных аналогов можно было сделать вывод о целесообразности проектирования и разработки программной системы. Результаты, полученные в ходе выполнения технико-экономического обоснования только подтвердили данный вывод.

Разработано программное средство, целевой платформой которого является веб-приложение и которое поддерживает следующие функции:
\begin{itemize}
	\item отображение расписания занятий;
	\item создание и управление индивидуальными заданиями и материалами;
	\item отображение прогресса выполнения заданий по предметам (для студентов) и прогресса выполнения заданий по всем людям (для преподавателей);
	\item отправка результатов выполнения заданий;
	\item возможность оценивания результатов выполнения преподавателем;
	\item обмен сообщениями между пользователями системы;
	\item подтверждение студентов и преподавателей с помощью введения специальной роли администратора факультета.
\end{itemize}

Следующая основная цель -- внедрение и популяризация программного средства среди преподавателей (в первую очередь) и студентов. Параллельно с этим будет производиться дальнейшая его разработка. Будет внедрена поддержка сессии: экзаменов и зачётов, -- с формированием виртуальной зачетки студента. Кроме того, будут внедряться новые функции, например: автоматизация проверки посещаемости с использованием геолокации и QR-кодов, автоматизация проверки результатов выполнения заданий по формальным признакам, создание расписания занятий для всех групп, поддержка различных типов занятий, включая разовые и дополнительные консультации.
