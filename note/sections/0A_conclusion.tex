\sectioncentered*{Заключение}
\addcontentsline{toc}{section}{Заключение}

В ходе разработки были исследованы особенности программных средств категории конструкторов сайтов, парадигмы разработки на языке программных средств на языке программирования JavaScript, в частности функциональная парадигма.

Был произведен анализ предметной области, выбраны основные технологии и инструменты для разработки, разработаны функциональные требования к программному средству. Позднее на их основании было произведено проектирование разрабатываемого программного средства.

В процессе проектирования был принят ряд решений об использовании определенных паттернов проектирования~\cite{wiki_design_patterns}, был разработан основной алгоритм работы программного средства и определена его архитектура, в качестве которой выступает SPA~\cite{wiki_spa}.

В процессе разработки программное средство было реализовано в качестве так называемого <<апплета>>~\cite{wiki_applet}

С целью проверки соответствия функциональным требованиям и корректности работы программного средства в целом была разработана серия тестовых сценариев, которые впоследствии были успешно пройдены: все получаемые результаты работы программного средства полностью совпадали с ожидаемыми.

Результаты тестирования свидетельствуют о том, что разработанное программное средство полностью соответствует требованиям и будет функционировать корректно, согласно предъявляемым к нему требованиям.

Помимо вышеперечисленного, разработке данного программного средства было дано технико-экономическое обоснование, включающее в себя расчет экономической эффективности разработки и реализации программного средства создания веб-приложений. Целесообразность разработки была подтверждена, а расчеты показали, что вложенные в проект финансовые инвестиции, окупятся за 14 месяцев.

По итогам разработки и тестирования можно сделать вывод, что разработанное программное средство разработано и функционирует в соответствии с предъявляемыми требованиями. Важным является тот факт, что приложение легко расширяется в контексте функциональности, а также предоствляет разработчикам возможность самим определять необходимую надстраиваемую поверх базовой функциональность.

Подытожив, можно сделать вывод, что разработанное программное средство соответствует задачам, поставленным в дипломном проектировании. Отсюда следует, что программное средство может быть поставлено на рынок. Цели дипломного проекта можно считать достигнутыми.