\subsection{Cборщик проекта}
\label{sec:development:bundler}

Инструменты сборки стали неотъемлемой частью веб-разработки, в основном из-за возрастающей сложности JS-приложений. Бандлеры позволяют нам упаковывать, компилировать, организовывать множество ресурсов и библиотек, необходимых для современного веб-проекта.

В ходе процесса разработки в качестве сборщика проекта был использован Webpack.

Webpack - сборщик проекта, который помогает автоматизировать второстепенные задачи веб-разработки и чаще всего используется для оперирования файлами и папками внутри проекта, преобразования таблиц стилей, написанных с применением синтаксиса SASS, LESS и др. в формат стандартных таблиц стилей CSS. Его также часто используют для выполнения задачи минификации кода проекта при помощи специальных плагинов – оптимизации в один или несколько файлов, что исключительно положительно сказывается на конечной производительности проекта.

Помимо этого, Webpack может путем анализа кода проекта удалять из него неиспользуемые части, что может привести к значительному уменьшению минифицированного варианта проекта. Из всего вышесказанного можно сделать вывод о том, что Webpack – мощный инструмент, который существенно облегчает процесс разработки проекта.

Webpack, мощный бандлер с открытым исходным кодом, который может обрабатывать огромное количество различных задач и является одним из самых мощных и гибких инструментов для сборки frontend. 

Плюсы:

\begin{itemize}
    \item великолепен для работы с одностраничными приложениями;
    \item воспринимает как require()- так и import-синтаксисы модуля;
    \item позволяет осуществлять продвинутое разделение кода;
    \item hot reload для более быстрой разработки с помощью React, Vue.js и подобных  фреймворков;
    \item наиболее популярный инструмент разработки по версии обзора JS в 2016 году.
\end{itemize}

Минусы:

\begin{itemize}
    \item не подойдет для новичков;
    \item работа с файлами CSS, картинками и другими не JS ресурсами по началу сбивает с толку;
    \item очень много изменений, большинство гайдов 2016 уже устарели;
\end{itemize}
