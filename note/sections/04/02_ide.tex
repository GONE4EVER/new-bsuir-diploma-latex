\subsection{Среда разработки Visual Studio Code}
\label{sec:development:ide}

Visual Studio Code — программное средство, разработанный Microsoft для Windows, Linux и macOS. Позиционируется как «лёгкий» редактор кода для кроссплатформенной разработки веб- и облачных приложений. Включает в себя отладчик, инструменты для работы с Git, подсветку синтаксиса, IntelliSense и средства для рефакторинга. Имеет широкие возможности для кастомизации: пользовательские темы, сочетания клавиш и файлы конфигурации. 

Распространяется бесплатно, разрабатывается как программное обеспечение с открытым исходным кодом, но готовые сборки распространяются под проприетарной лицензией.

Visual Studio Code основан на Electron — фреймворк, позволяющий с использованием Node.js разрабатывать настольные приложения, которые работают на движке Blink. Несмотря на то, что редактор основан на Electron, он не использует редактор Atom. Вместо него реализуется веб-редактор Monaco,разработанный для Visual Studio Online.

Visual Studio Code — это редактор исходного кода. Он поддерживает ряд языков программирования, подсветку синтаксиса, IntelliSense, рефакторинг, отладку, навигацию по коду, поддержку Git и другие возможности. Многие возможности Visual Studio Code не доступны через графический интерфейс, зачастую они используются через палитру команд или JSON файлы (например, пользовательские настройки). Палитра команд представляет собой подобие командной строки, которая вызывается сочетанием клавиш.

Visual Studio также позволяет заменять кодовую страницу при сохранении документа, символы перевода строки и язык программирования текущего документа.

С 2018 года появилось расширение Python для Visual Studio Code с открытым исходным кодом. Оно предоставляет разработчикам широкие возможности для редактирования, отладки и тестирования кода.

На март 2019 года посредством встроенного в продукт пользовательского интерфейса можно загрузить и установить несколько тысяч расширений только в категории «programming languages» (языки программирования).