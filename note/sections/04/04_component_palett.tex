\subsection{Разработка палитры компонентов}
\label{sec:development:component_palett}

Для инициализации компонента палитры были использованы компоненты библиотеки Webix. Для отображения разных вкладок на панели использвался компонент TabView, а для отображения данных и источника <<перетягивания>> комопннетов использовался компонент DataView, так как имеет встроенные возможности для работы с алгоритмом Drag-n-drop. 

\subsubsection{}Способ предсоставления компонентов
\

Каждой вкладке соответствует свой DataView, отображающий дозированные согласно названию вкладки данные, управляет отображением же TabView. Далее будет представлен основной код палитры компонентов.

\begin{lstlisting}
import getAllTabConfig from './palletTabs/all';
import { getPresetsTabConfig } from './palletTabs/presetsTab';

const getDefaultComponentPalletLayout = dataContent => ({
  rows: [
    {
      view: 'search',
      placeholder: 'Search 1,000,000 images...'
    },
    {
      view: 'tabview',
      cells: [
        {
          header: 'All',
          body: getAllTabConfig(dataContent)
        },
        {
          header: 'Presets',
          body: getPresetsTabConfig()
        },
        ...componentContainers(dataContent)
      ]
    }
  ]
});
\end{lstlisting}

Содержимое массива, возвращаемого функцией componentContainers являются DataView компоненты, каждый и который соответствует сожержанием своей вкладке.

\begin{lstlisting}
const DATAVIEW_TEMPLATE = '<span class="webix-icon fa fa-#icon#"></span>  #element#';
const LAYOUT_DATAVIEW_CONFIG = 'pallet:dataview:layouts';
const COMPONENTS_DATAVIEW_CONFIG = 'pallet:dataview:components';
const CUECARDS_DATAVIEW_CONFIG = 'pallet:dataview:cuecards';

const componentContainers = data => [{
  header: 'Layouts',
  body: {
    view: 'dataview',
    template: DATAVIEW_TEMPLATE,
    id: LAYOUT_DATAVIEW_CONFIG,
    xCount: 3,
    drag: 'source',
    minHeight: 200,
    data: data.layouts
  }
},
{
  header: 'Components',
  body: {
    view: 'dataview',
    template: DATAVIEW_TEMPLATE,
    id: COMPONENTS_DATAVIEW_CONFIG,
    xCount: 3,
    drag: 'source',
    minHeight: 200,
    data: data.components
  }
},
{
  header: 'Cue cards',
  body: {
    view: 'dataview',
    template: DATAVIEW_TEMPLATE,
    id: CUECARDS_DATAVIEW_CONFIG,
    xCount: 3,
    drag: 'source',
    minHeight: 200,
    data: data.templates.map(element => ({
      ...element,
      template: element.template
    }))
  }
}]
\end{lstlisting}

Каждый DataView компонент отвечает за отображение своей части данных, а указание свойства <<drag: 'source' предоставляет возможность использовать компонент как источник данных (пункта <<тащи>> или <<drag>>) для реализации алгоритма Drag-n-drop.

Объект конфигураций самого компонента выглядит следующим образом:

\begin{lstlisting}
import getDefaultComponentPalletLayout from './palletLayoutConfig';
import { onPresetLoad } from './palletTabs/presetsTab';
import ComponentPallet from '../layouts/interfaces/ComponentPallet';

const componentPalletConfig: ComponentPallet = {
  name: 'AppOrchid-Component-Pallet',
  defaults: {
    // defaults here

  },
  $init() {
    // initialization here
  },
  _getComponentViewConfig() {
    if (!this.config.componentView || typeof this.config.componentView !== 'object') {
      return this.config.defaultComponentView;
    }

    return this.config.componentView;
  },
  _getComponentView() {
    return this.queryView({ ...this.config.defaultComponentView });
  }
};
\end{lstlisting}

Загрузка содержимого DataView компонентов -- преопределенных конфигураций графических компонентов, осуществляется самим Component-Pallet при его инициализации путем, регламентированным библиотекой Webix.

\begin{lstlisting}
$init() {
  this.$ready.push(
    () => {
    webix.ajax().get(this.config.widgets)
      .then(res => res.json())
      .then(res => ({
        layouts: res.layouts.map(item => ({ ...item, widgetType: 'layout' }),
        components: res.components.map(item => ({ ...item, widgetType:'component' })),
        templates: res.templates.map(item => ({ ...item, widgetType: 'card'}))
      }))
      .then((res) => {
        this.addView({
          ...getDefaultComponentPalletLayout(res)
        });
      })
      .catch((e) => {
        console.error(e);
      });
    },
    () => {
      if (!this.config.gridId) { return; }
      let loadPreset = onPresetLoad.bind(this);
      this.attachEvent('onPresetLoad', loadPreset);
    }
  );
}
\end{lstlisting}

Применение кастомной конфигурации к базовому компоненту осуществляется путем применения метода protoUI глобального объекта webix к конфигурации нашего компонента Component-Pallet и базовым классам, предоставляемым библиотекой Webix:
\begin{itemize}
    \item webix.IdSpace -- для изоляции содержимого;
    \item webix.ui.layout -- для доступа к функциональности базового лэйаута.
\end{itemize}

\begin{lstlisting}
webix.protoUI(
  componentPalletConfig,
  webix.IdSpace,
  webix.ui.layout
);
\end{lstlisting}

По умолчанию предполагается, что в получаемом в результате запроса JavaScript-объекте должны быть поля: layouts, components и card -- три вида лэйаутов. Однако, эти параметры могут быть изменены разработчиком, интегрирующим данное програмнное средство в свой проект.

\subsubsection{}Пресеты
\

В палитре компонентов, как указывалось ранее, доступна возможность загрузки и применения пресетов -- предопределенных наборов компонентов, которые были созданы и предоставлены разработчиками или, например, другими пользователями.

Доступ к пресетам осуществляется через соответствующую вкладку <<Presets>>. Лист пресетов получает их конфигурации из объекта, реализующего доступ к данным в библиотеке Webix -- DataCollection.

\begin{lstlisting}
{
  id: getPresetsListId(),
  borderless: true,
  data: webix.storage.local.get('collection') || [{ name: 'No presetsfound' }],
  view: 'list',
  select: true,
  template: `#name#${DELETE_PRESET_ICON_TEMPLATE}`,
  on: {
      onItemDblClick(id): void {
      const topParent = this.getTopParentView();
      topParent.callEvent('onPresetLoad', [this.getItem(id)]);
      }
  },
  onClick: {
    'fa-trash-o': function (ev, id): void {
      const self = this;
      webix.confirm({
        text: 'Item data will be lost. <br/> Are you sure?',
        ok: 'Yes',
        cancel: 'Cancel',
        callback: (res) => {
          removePreset.apply(self, [res, id]);
        }
      });
    }
  }
}
\end{lstlisting}

Сохранение текущей конфигурации осуществляется путем сериализации содержимого грида. Для этого вызывается метод объекта грида getSerializedCollection, логика которого будет рассмотрена позднее.

\begin{lstlisting}
const topParent = $$(this.getTopParentView().config.master)
.getTopParentView();

const gridId = $$((topParent.config).gridId);

const newElement = savePreset(this.getValue(), $$(gridId)
  .getSerializedCollection());
this.getTopParentView().hide();

if (newElement) {
  topParent.$$(getPresetsListId()).parse(newElement);
}
\end{lstlisting}

Применение пресета из списка доступных осуществляется в несколько этапов. 
Первым является вызов события onPresetLoad, на срабатывание которого вызывается колбэк с соответствующим именем, в котором осуществляется первичная проверка данных и дальнейшая делегация выполнения применения пресета.

\begin{lstlisting}
function onPresetLoad(preset) {
  const grid = ($$(this.config.gridId));

  if (preset.content) {
    let arr = $$(this.config.gridId)
      .getChildViews();

    for (let i = arr.length - 1; i > 0; i--) {
      grid.removeView(arr[i]);
    }

    webix.delay(() => {
      loadPreset(preset.content, grid);
    });

    webix.message({
      text: `Preset ${preset.name} loaded`,
      expire: 1000
    });
  } else {
    webix.message({
      text: 'No data provided',
      type: 'error'
    });
  }
}
\end{lstlisting}

Вызываемая внутри функция loadPreset вызывает функцию saveAndRender для каждого объекта, которому соответствует компонент.

\begin{lstlisting}  
const loadPreset = (collection: object, grid: ComponentsLayout)=> {
  Object.keys(collection)
    .forEach((presetId) => {
    saveAndRender(collection[presetId], grid);
  });
};    
\end{lstlisting}

Метод saveAndRender в свою очередь добавляет нужные позиционные свойства компонентам, которые будут помещаться на грид и вызывает у грида метод рендера. В качестве параметра туда передается конфиг компонента и его будущие координаты.

\begin{lstlisting}
    
const saveAndRender = (source: any, grid: ComponentsLayout) => {
  const id = webix.uid();
  let item = { id, ...source };
  
  if (!item.parentId) {
    const width = item.width || 0;
    const height = item.height || 0;
  
    const offsetX = Number(item.left + (width / 2));
    const offsetY = Number(item.top + (height / 2));
  
    grid._addMovableElement(item, {
      offsetX,
      offsetY
    });
  }
  return item;
};
\end{lstlisting}

Удаление пресета из списка осуществляется путем удаления пресета из коллекции, к которой <<привязан>> список пресетов. Путем простой фильтрации по условию неравенства уникальному свойству удаляемого пресета и осуществляется процесс удаления.

\begin{lstlisting}
function removePreset(result: any, id: string | number) {
  if (result) {
    let presetsCollection = webix.storage.local.get('collection');
    if (!presetsCollection || !presetsCollection.length) {
      return;
    }
    presetsCollection.filter(element => element.name !== this.getItem(id).name);

    this.remove(id);

    webix.storage.local.put('collection', presetsCollection);
  }
}
\end{lstlisting}

Резюмируя, можно сделать вывод, что был разработан гибкий инструмент для использования предопределенных наборов компонентов, предоставляющий функциональность как добавления и применения пресетов, так и их удаления.