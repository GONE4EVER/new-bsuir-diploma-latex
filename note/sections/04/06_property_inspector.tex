\subsection{Разработка инспектора свойств}
\label{sec:development:property_inspector}

Инспектор свойств является панелью, содержащей множество вкладок, поэтому были использованы такие базовые графические компоненты, предоставляемые библиотекой Webix, как:

\begin{itemize}
    \item TabView -- для дозирования количества одновременно отображаемой информации путем предоставления доступа к ее частям посредством разделения на страницы;
    \item Accordion -- для группирования отображаемой информации в логически или технически обусловленные группы свойств.
\end{itemize}

Конфиг компонента представляет собой набор <<страниц>>, реализованный посредством компонента TabView. Каждая страница содержит в себе компонент Accordion, позволяющий манипулировать отображением собственного содержимого.

Компонент Accordion, использующийся с целью отображения свойств выделенных компонентов, в свою очередь использует специальный компонент библиотеки Webix, называющийся <<property>>, который был специально создан для подобных целей.

Метод onAfterEditStop вызывается по окончании пользователем ввода данных в поле свойства. Здесь осуществляются проверки введенных данных на валидность и их дальнейшая обработка.

На случай ввода неверных данных предусмотрены дополнительные обработчики и средства вывода информации, с целью уведомить пользователя об ошибке.

Основным инструментом для работы с компонентами является панель свойств, отображающая свойства выделенного компонента.

Чтобы после выделения пользователем компонента его свойства отображались во вкладке свойства инспектора свойств необходимо было добавить в его метод инициализации соовтествующий обработчик события.

При срабатывании события onMovableElementSelect происходит копирование свойств компонента с помощью метода библиотеки. Функции showColumnsForm и hideColumnsForm отвечают за управление отображением дополнительной вкладки свойств, доступной лишь после выделения особого компонента, предоставляемого библиотекой Webix, нужнающегося в дополнительной конфигурации -- Datatable.

Данная особенность обусловлена особой струкурой компонента и соответсвующего ему DOM-элемента.

Помимо всего вышеперечисленного, инспектор свойств предоставляет возможность поиска свойства как по ключу, так и по значению, что значительно упрощает работу с комплексными компонентами, содержащими большое количесво свойств.

Для этого в самом верху инспектора свойств расположена панель ввода с соовтествующей иконкой.

Функция getInputValue содержит внутри себя логику, реализующие поиск данных по ключу и значению.

По окончании выполнения вышеуказанных в функции getInputValue действий, все найденные свойства добавятся в содержимое компонента, отображающего результаты поиска в виде списка. Вкладка с результатами поиска будет открыта автоматически, что определенно благоприятно скажется на UX.

Функциональность самого списка результатов поиска не ограничивается одним лишь отображением самих результатов поиска -- при нажатии на один из отображаемых результатов система автоматически отобразит пользователю нужную вкладку и, более того, сразу откроет поле с искомым свойтсвом в режиме редактирования.

Ниже будет предоставлен фрагмент кода с конфигом данного компонента, где будут видны некоторые детали реализации.

Подобная реализация также определенно благоприятно скажется на UX, ведь в перспективе пользователю необходимо будет совершать значительно меньше действий для достижения желаемого результата.