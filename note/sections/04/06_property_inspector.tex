\subsection{Разработка инспектора свойств}
\label{sec:development:property_inspector}

Инспектор свойств является панелью, содержащей множество вкладок, поэтому были использованы такие базовые графические компоненты, предоставляемые библиотекой Webix, как:

\begin{itemize}
    \item TabView - для дозирования количества одновременно отображаемой информации путем предоставления доступа к ее частям посредством разделения на страницы;
    \item Accordion - для группирования отображаемой информации в логически или технически обусловленные группы свойств.
\end{itemize}

Конфиг компонента представляет собой набор <<страниц>>, реализованный посредством компонента TabView. Каждая страница содержит в себе компонент Accordion, позволяющий манипулировать отображением собственного содержимого.

\begin{lstlisting}
this.addView({
  rows: [
    undoRedoConfig,
    { ...getSearchPanel() },
    {
      id: webix.uid(),
      borderless: true,
      view: 'tabview',
      cells: [
        {
          header: 'Data',
          ...getAccDataConfig()
        }
        {
          header: 'Style',
          ...getAccStyleConfig(this)
        }
        {
          header: 'Code',
          ...getElementCodeConfig()
        }
      ]
    }
  ]
})
\end{lstlisting}

Компонент Accordion, использующийся с целью отображения свойств выделенных компонентов, в свою очередь использует специальный компонент библиотеки Webix, называющийся <<property>>, который был специально создан для подобных целей.

Метод onAfterEditStop вызывается по окончании пользователем ввода данных в поле свойства. Здесь осуществляются проверки введенных данных на валидность и их дальнейшая обработка.

На случай ввода неверных данных предусмотрены дополнительные обработчики и средства вывода информации, с целью уведомить пользователя об ошибке.

\begin{lstlisting}
{
  id: getSettingsId(),
  view: 'property',
  onContext: {},
  elements: getElements(),
  autoheight: true,
  nameWidth: 120,
  on: {
    onAfterEditStop(state, editor): void {
      if (Number.isInteger(Number(state.value))) {
        state.value = Number(state.value);
      }

      if (Number(state.value) !== state.old) {
        let view = <any>this.getTopParentView()
          .getLastActiveComponent();

        if (view) {
          if (editor.id === 'url') {
            view.clearAll();
            view.config.data = null;
          } else if (editor.id === 'top' || editor.id === 'left') {
            view.getParentView().config[editor.id] = state.value;
            view.getParentView()
                .resize();
          }

          // visible component editing
          view.config[editor.id] = state.value;

          if (view.setTemplate) {
            view.setTemplate();
          }
          if (view.refresh) {
            view.refresh();
          }

          view.resize();
          view.callEvent('onBlur');
          view.callEvent('onFocus', [view]);
        } else {
          webix.message({
            text: 'Attempt to change property of not existing item',
            type: 'error',
            expire: 2000
          });
        }
      }
    }
  }
}
\end{lstlisting}

Основным инструментом для работы с компонентами является панель свойств, отображающая свойства выделенного компонента.

\begin{lstlisting}
definePropertyTable(viewConfig) {
  let elements = [];
  let propertyTable = this.$$(getSettingsId());

  this.$$(getAccStyleId()).config.data = viewConfig;

  const elementCodeView = this.$$(getElementCodeId());

  if (viewConfig) {
      if (viewConfig.view === 'datatable') {
          elements = getDatatableElements();
      } else {
        Object.entries(viewConfig)
          .forEach(([key]) => {
            if (
              key === 'id'
              || key === '_inner'
              || key === 'template'
            ) {
              return;
            }

            let type = 'text';
            if (key.search(/color/gi) !== -1) {
              type = 'color';
            }

            elements.push({
              id: key,
              label: key,
              type
            });
        });
      }
  }

  elementCodeView.setValue(`${getPrettyJson(viewConfig)}`);
  elementCodeView.refresh();

  propertyTable.define('elements', elements);

  let parsedItem = getParsedConfig(viewConfig);

  propertyTable.setValues(parsedItem);

  propertyTable.resize();
}
\end{lstlisting}

Чтобы после выделения пользователем компонента его свойства отображались во вкладке свойства инспектора свойств необходимо было добавить в его метод инициализации соовтествующий обработчик события.

\begin{lstlisting}
() => {
  if (!this.config.gridId) { return; }
  eventHandlerInit(this.config.gridId);
  const grid = $$(this.config.gridId);

  grid.attachEvent('onMovableElementSelect', target) =>{
    focusEventHandler(target);
    this.setLastActiveComponent(target);
    let configCopy = null;

    try {
      configCopy = webix.copy(target.config);
    } catch (e) {
      configCopy = webix.clone(target.config);
    }
    this.definePropertyTable(configCopy);

    if (target.config.view === 'datatable') {
      showColumnsForm(this, target);
    } else {
      hideColumnsForm(this);
    }
  });
}
\end{lstlisting}

При срабатывании события onMovableElementSelect происходит копирование свойств компонента с помощью метода библиотеки. Функции showColumnsForm и hideColumnsForm отвечают за управление отображением дополнительной вкладки свойств, доступной лишь после выделения особого компонента, предоставляемого библиотекой Webix, нужнающегося в дополнительной конфигурации - Datatable.

Данная особенность обусловлена особой струкурой компонента и соответсвующего ему DOM-элемента.

Помимо всего вышеперечисленного, инспектор свойств предоставляет возможность поиска свойства как по ключу, так и по значению, что значительно упрощает работу с комплексными компонентами, содержащими большое количесво свойств.

Для этого в самом верху инспектора свойств расположена панель ввода с соовтествующей иконкой.

Далее будет представлен код, описывающий логику данного компонента.

\begin{lstlisting}
const tabviewSearch = {
  id: getSearchPanelId(),
  view: 'search',
  hotkey: 'enter',
  placeholder: 'Search any property',
  on: {
    onChange() {
      getInputValue.call(this);
    },
    onFocus() {
      // blocking events in order to prevent starting search
      this.blockEvent();
      this.getNode().children[0].children[0].select();
      this.unblockEvent();
    },
    onEnter() {
      getInputValue.call(this);
    }
  }
};
\end{lstlisting}

Секция on содержит обработчики одноименных событий. Свойство hotkey указывает горячую клавишу для вызова события onEnter, которое осуществляет вызов функции getInputValue с контектом объекта компонента поиска.

Функция getInputValue инкапсулирует функциональность поиска данных по ключу и значению. Ее код преставлен ниже.

\begin{lstlisting}
function getInputValue(): void {
  const list = <webix.ui.list>this.getTopParentView().$$(getSearchListId());
  const val: string = this.getTopParentView().$$(getSearchPanelId()).getValue().toLowerCase()
    .trim();
  const store: object = getParsedConfig(this.getTopParentView().$$(getAccStyleId()).config.data);

  list.clearAll();

  if (store && val) {
    this.getTopParentView().$$(getSearchListId()).clearAll();

    Object.entries(store).forEach(([key, value]) => {
      if (JSON.stringify(value) !== undefined
        && (key.toLowerCase().indexOf(val) !== -1
          || JSON.stringify(value).indexOf(val) !== -1)
      ) {
        if (typeof value === 'string') {
          value = value.trim();
        }
        list.parse({ key, value }, '');
      }
    });
  }

  this.blur();
  list.getParentView().getParentView().show();
  list.getParentView().define({ collapsed: false });
  list.getParentView().refresh();
}
\end{lstlisting}

По окончании выполнения вышеуказанных в функции getInputValue действий, все найденные свойства добавятся в содержимое компонента, отображающего результаты поиска в виде списка. Вкладка с результатами поиска будет открыта автоматически, что определенно благоприятно скажется на UX.

Функциональность самого списка результатов поиска не ограничивается одним лишь отображением самих результатов поиска - при нажатии на один из отображаемых результатов система автоматически отобразит пользователю нужную вкладку и, более того, сразу откроет поле с искомым свойтсвом в режиме редактирования.

Ниже будет предоставлен фрагмент кода с конфигом данного компонента, где будут видны некоторые детали реализации.

\begin{lstlisting}
const searchListId = 'AC:settings:searchList';

const getSearchListId = (): string => searchListId;

const searchListConfig = {
  id: getSearchListId(),
  template: '{#key#: #value#}',
  select: true,
  view: 'list',
  on: {
    onItemClick(itemId): void {
      const accordion = this.getTopParentView().$$(getScrollViewId()).getParentView().getParentView();
      accordion.define('collapsed', false);

      let key = this.getItem(itemId).key;

      if (key !== 'id') {
        this.getTopParentView().$$(getSettingsId()).edit(this.getItem(itemId).key);
      }
    }
  },
  data: []
}
\end{lstlisting}

Подобная реализация также определенно благоприятно скажется на UX, ведь в перспективе пользователю необходимо будет совершать значительно меньше действий для достижения желаемого результата.