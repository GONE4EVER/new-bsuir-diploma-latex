\sectioncentered*{Определения и сокращения}
\label{sec:definitions}

\emph{Апплет} -- это несамостоятельный компонент программного обеспечения, работающий в контексте другого, полновесного приложения~\cite{wiki_applet}.

\emph{Асинхронность (Асинхронизм)} -- не совпадение с чем-либо во времени; неодномоментность, неодновременность, несинхронность — характеризует процессы, не совпадающие во времени.

\emph{Грид} -- это двумерная система компоновки основанная на сетке, цель которой заключается в том чтобы полностью изменить способ проектирования пользовательских интерфейсов основанных на сетке~\cite{grid}.

\emph{Колбэк} -- функция обратного вызова в программировании — передача исполняемого кода в качестве одного из параметров другого кода. Обратный вызов позволяет в функции исполнять код, который задаётся в аргументах при её вызове. Этот код может быть определён в других контекстах программного кода и быть недоступным для прямого вызова из этой функции. Некоторые алгоритмические задачи в качестве своих входных данных имеют не только числа или объекты, но и действия (алгоритмы), которые естественным образом задаются как обратные вызовы (сокращение от англ. Сallback: all — вызов, англ. back — обратный)~\cite{wiki_callback}.

\emph{Лэйаут} -- структурированное отображение информации на плоскость~\cite{wiki_layout}.

\emph{Сериализация} -- процесс перевода какой-либо структуры данных в последовательность битов. Обратной к операции сериализации является операция десериализации (структуризации) — восстановление начального состояния структуры данных из битовой последовательности~\cite{wiki_serialization}.

AJAX (сокращение от англ. Asynchronous JavaScript and XML -- асинхронный JavaScript и XML) -- подход к построению интерактивных пользовательских интерфейсов веб-приложений, заключающийся в <<фоновом>> обмене данными браузера с веб-сервером. В результате, при обновлении данных веб-страница не перезагружается полностью, и веб-приложения становятся быстрее и удобнее~\cite{wiki_ajax}.

\emph{Babel} -- это бесплатный компилятор JavaScript с открытым исходным кодом и настраиваемый транспилер, используемый в веб-разработке. Babel позволяет разработчикам программного обеспечения писать исходный код на предпочтительном языке программирования или языке разметки и переводить его Babel на JavaScript, язык, понятный современным веб-браузерам~\cite{babel}.

CMS (сокращение от англ. Content Management System -- <<система управления контентом сайта>> или просто <<система управления сайтом>>) -- программное обеспечение, которое позволяет разрабатывать и поддерживать динамические информационные web-сайты~\cite{cms}.

DOM (от англ. Document Object Model — «объектная модель документа») -- это не зависящий от платформы и языка программный интерфейс, позволяющий программам и скриптам получить доступ к содержимому HTML-, XHTML- и XML-документов, а также изменять содержимое, структуру и оформление таких документов~\cite{wiki_dom}.

\emph{Drag-and-drop} (в переводе с английского -- тащи-и-бросай; Бери-и-Брось) -- способ оперирования элементами интерфейса в интерфейсах пользователя (как графическим, так и текстовым, где элементы GUI реализованы при помощи псевдографики) при помощи манипулятора «мышь» или сенсорного экрана~\cite{wiki_dnd}.

\emph{Git} -- распределённая система контроля версий, которая даёт возможность разработчикам отслеживать изменения в файлах и работать совместно с другими разработчиками. Она была разработана в 2005 году Линусом Торвальдсом, создателем Linux, для того, чтобы другие разработчики могли вносить свой вклад в ядро Linux. Git известен своей скоростью, простым дизайном, поддержкой нелинейной разработки, полной децентрализацией и возможностью эффективно работать с большими проектами~\cite{git}.

UI --  интерфейс, обеспечивающий передачу информации между пользователем-человеком и программно-аппаратными компонентами компьютерной системы (сокращение от англ.  User Interface -- <<пользовательский интерфейс>>)~\cite{wiki_ui}.

REST -- архитектурный стиль взаимодействия компонентов распределенного приложения (сокращение от англ. Representational State Transfer -- <<передача состояния представления>>)~\cite{wiki_rest}.

SPA (сокращение от англ. Single Page Application -- одностраничное приложение) -- это веб-приложение или веб-сайт, использующий единственный HTML-документ как оболочку для всех веб-страниц и организующий взаимодействие с пользователем через динамически подгружаемые HTML, CSS, JavaScript, обычно посредством AJAX~\cite{wiki_spa}.

\emph{Trial (триал)} -- Условно бесплатное программное обеспечение (программное обеспечение с безвозмездным (или возмездным при определённых условиях) использованием)~\cite{wiki_trial}.

UX (сокращение от англ. User Experience -- «опыт пользователя») -- это то, какой опыт/впечатление получает пользователь от взаимодействия с вашим интерфейсом. Удается ли ему достичь цели и насколько просто или сложно это сделать~\cite{habr_ux}.

\sectioncentered*{Введение}
\addcontentsline{toc}{section}{Введение}
\label{sec:introduction}

С появлением Web-технологиий компьютер начинают использовать совершенно новые слои населения Земли. Можно выделить две наиболее характерные группы, находящиеся на разных социальных полюсах, которые были стремительно вовлечены в новую технологию, возможно, даже помимо их собственного желания. С одной стороны, это были представители элитарных групп общества: руководители крупных организаций, президенты банков, топ- менеджеры, влиятельные государственные чиновники и т.д. С другой стороны, это были представители широчайших слоев населения: домохозяйки, пенсионеры, дети.

При появлении технологии Web компьютеры повернулись лицом к этим двум совершенно противоположным категориям потенциальных пользователей. Элиту объединяла одна черта -- в силу высочайшей ответственности и практически стопроцентной занятости <<большие люди>> никогда не пользовались компьютером; типичной была ситуация, когда с компьютером работал секретарь. В какой-то момент времени они поняли, что компьютер им может быть полезен, что они могут результативно использовать то небольшое время, которое можно выделить на работу за компьютером. Они вдруг поняли, что компьютер -- это не просто модная и дорогая игрушка, но инструмент получения актуальной информации для бизнеса. При этом им не нужно было тратить сколько-нибудь заметного времени, чтобы освоить технологию работы с компьютером (по сравнению с тем, как это было раньше).

Спектр социальных групп, подключающихся к сети Интернет и ищущих информацию в WWW, все время расширяется за счет пользователей, не относящихся к категории специалистов в области информационных технологий. Это врачи, строители, историки, юристы, финансисты, спортсмены, путешественники, священнослужители, артисты, писатели, художники. Список можно продолжать бесконечно. Любой, кто ощутил полезность и незаменимость Сети для своей профессиональной деятельности или увлечений, присоединяется к огромной армии потребителей информации во <<Всемирной Паутине>>.

Web-технология полностью перевернула наши представления о работе с информацией, да и с компьютером вообще. Оказалось, что традиционные параметры развития вычислительной техники -- производительность, пропускная способность,емкость запоминающих устройств не учитывали главного <<узкого места>> системы -- интерфейса взаимодействия с человеком. Устаревший механизм взаимодействия человека с информационной системой сдерживал внедрение новых технологий и уменьшал выгоду от их применения. И только когда интерфейс между человеком и компьютером был упрощен до естественности восприятия обычным человеком, последовал беспрецедентный взрыв интереса к возможностям вычислительной техники.

С развитием технологий гипертекстовой разметки в Интернете стало появляться всё больше сайтов, тематика которых была совершенно различной -- от сайтов крупных компаний, повествующих об успехах компании и её провалах, до сайтов маленьких фирм, предлагающих посетить их офисы в пределах одного города.

Развитие Интернет-технологий послужило толчком к появлению новой ветки в Интернете -- Интернет-форумов. Стали появляться сайты, и даже целые порталы, на которых люди со всех уголков планеты могут общаться, получать ответы на любые вопросы и, даже, заключать деловые сделки.

Основные идеи современной информационной технологии базируются на концепции, согласно которой данные должны быть организованы в базы данных, с целью адекватного отображения изменяющегося реального мира и удовлетворения информационных потребностей пользователей.

Любая информационная система представляет собой программный комплекс, функции которого состоят в поддержке надежного хранения информации в памяти компьютера, выполнении специфических для данного приложения преобразований информации и/или вычислений, предоставлении пользователям удобного и легко осваиваемого интерфейса. С развитием и распространением сети Интернет информационные системы стали более интерактивными, масштабируемыми и доступными обычным пользователям.

Необходимость систем управления для владельцев сайтов начала проявляться в тот момент, когда количество материалов на веб-сайтах начало стремительно расти. Это привело к тому, что традиционные <<ручные>> технологии разработки и поддержки сайтов, когда сайт состоял из статических страниц и набора дополнительных специализированных скриптов, стали не успевать за быстро меняющимися условиями бизнеса. Ввод данных на сайт требовал (как минимум) знания технологий HTML/CSS верстки, изменения структуры сайтов были сопряжены с каскадным изменением большого количества взаимосвязанных страниц. Различные автоматизированные механизмы, вроде гостевых книг и новостных лент, внедренные на сайтах как отдельные скрипты и, как правило, написанные разными специалистами, перестали удовлетворять требованиям безопасности. На многих сайтах стали появляться коктейли из разных технологий и подходов к разработке, поэтому возникла потребность в стандартизации программных решений, в разделении дизайна и содержимого на две независимые составляющие. CMS действительно разделяют сайты на две составляющие: дизайн (внешний вид сайта в целом, отдельных страниц, конкретных блоков информации) и контент. Дизайн сайта как правило <<зашит>> в шаблоны и изменяется значительно реже, чем контент.

Система управления сайтами -- это программный комплекс, позволяющий автоматизировать процесс управления как сайтом в целом, так и сущностями в рамках сайта: макетами страниц, шаблонами вывода данных, структурой, информационным наполнением, пользователями и правами доступа, а также по возможности предоставляющий дополнительные сервисы: списки рассылки, ведение статистики, поиск, средства взаимодействия с пользователями и т. д.

Обычно системы обновления делятся на две части: внешнюю -- набор HTML-страниц, генерируемых при вызове страниц из браузера посетителя сайта и внутреннюю -- систему администрирования. Обе части обычно используют общее хранилище данных, в роли которого, как правило, выступает реляционная база данных (иногда встречаются другие виды хранилищ, например, XML-документы или даже текстовые файлы).

В данном дипломном проекте реализуется программное средство создания веб-приложений с помощью готовых графических компонентов, которое способно интегрироваться в другие программные решения, может использоваться как самими разработчиками для создания пользовательского интерфейса веб-приложения, так и пользователями этого веб-приложения для конфигурации пользовательского интерфейса согласно собственным предпочтениям.

Темой разрабатываемого проекта является <<Программное средство создания веб-приложений с помощью готовых графических компонентов>>. 

Перечень задач, поставленных для решения в течение преддипломной практики представлен ниже:

\begin{itemize}
    \item Анализ предмета, постановка задачи;
    \item Методы и модели, положенные в основу;
    \item Разработка структуры проекта. 
\end{itemize}

Целью дипломного проекта является разработка и реализация интегрируемого приложения, общающегося с сервером в сети Интернет с целью предоставления пользователям перечня доступных веб-компонентов. В первую очередь разрабатываемое программное средство предназначается для небольших проектов, в которые будет легко интегрироваться и при этом предоставлять возможности технологии, на которой написан.
