\section{Проектирование программного средства}
\label{sec:design}

Проектирование – процесс определения архитектуры, компонентов, интерфейсов и других характеристик системы или её части. Результатом проектирования является проект – целостная совокупность моделей, свойств или характеристик, описанных в форме, пригодной для последующей реализации системы. Оно, наряду с анализом требований, является частью большой стадии жизненного цикла системы, называемой определением системы. 

Проектирование системы направлено на представление системы, соответствующее предусмотренной цели, принципам и замыслам; оно включает оценку и принятие решений по выбору таких компонентов системы, которые отвечают её архитектуре и укладываются в предписанные ограничения.

\subsection{Проектирование алгоритма работы программы}
\label{sec:design:algorithm}

На рисунке 3.1 представлена схема алгоритма работы программного средства «Conference Viewer». 
На ней отражены такие возможные действия, доступные для выполнения, как рисование, подключение, отправка голосовых данных и отключение. 

В зависимости от конкретного действия, каждое содержит в себе собственные шаги реализации. Так при выборе действия «Рисование» возможна просто отрисовка, а также отрисовка с передачей данных другому подключённому пользователю. «Подключение» содержит в себе проверку введённых данных на корректность и сам процесс подключения. «Отправка голоса» осуществляется при условии подключения и нажатии кнопки для записи голосовых данных. «Отключение» заключается в отправке управляющего пакета для запроса отклчения с последующим принятием подтверждения и освобождения занимаемых ресурсов

